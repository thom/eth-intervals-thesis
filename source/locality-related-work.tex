%==============================================================================
% locality-related-work.tex
%==============================================================================

\chapter{Related Work}
\label{chap:locality-related-work}

\textcite{Acar2002} present a work-stealing algorithm that uses
locality information, and thus outperforms the standard work-stealing
algorithm on benchmarks. Each process maintains a queue of pointers to
threads that have affinity for it, and attempts to steal these
first. They also bound the number of cache misses for the
work-stealing algorithm, using a ``potential function'' argument.

\textcite{Squillante1993} proposes and compares various scheduling
algorithms which trade off load balancing and processor-cache
affinity. They use the detailed cache model to examine cache reload
time, as well as examining the effects of increased bus traffic. No
work-stealing is used since pool of tasks is shared.

\textcite{Philbin1996}: Thread scheduling for cache locality

Threading Building Blocks \cite{Contreras2008, Reinders2007} is
designed with caches in mind and works to limit the unnecessary
movement of tasks and data. When a task has to be passed to a
different processor core for execution, Threading Building Blocks
moves the task with the least likelihood of having data in the cache
for the processor core from which the task is stolen.

X10: \textcite{Charles2005} ``X10: an object-oriented approach to
non-uniform cluster computing'', \textcite{Saraswat2010} ``Report on
the programming language X10''

\textcite{Agarwal2008}: Static Detection of Place Locality and
Elimination of Runtime Checks

\textcite{Barik2009}: Efficient Optimization of Memory Accesses in
Parallel Programs

\textcite{Dinan2009}: Scalable Work Stealing

\textcite{Tam2007}: Thread clustering: sharing-aware scheduling on
SMP-CMP-SMT multiprocessors

\textcite{Yan2009}: Hierarchical Place Trees: A Portable Abstraction
for Task Parallelism and Data Movement

\textcite{Gaud2010}: Mely: Efficient Workstealing for Multicore
Event-Driven Systems \cite{Gaud2010}

\textcite{Guo2010}: SLAW: a Scalable Locality-aware Adaptive
Work-stealing Scheduler

\todo{Finish chapter ``Related Work''}


%%% Local Variables: 
%%% mode: latex
%%% TeX-master: "thesis"
%%% End: 