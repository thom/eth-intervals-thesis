%==============================================================================
% locality-related-work.tex
%==============================================================================

\chapter{Related Work}
\label{chap:locality-related-work}

Locality-aware scheduling is a popular area of research. In the
beginning, most research was done with shared pools of tasks.

\textcite{Squillante1993} explore the importance of using
processor-cache affinity information in shared-memory multiprocessor
scheduling. Their algorithms a shared pool of tasks and trade off load
balancing and processor-cache affinity.

\textcite{Philbin1996} use fine-grained threads to decompose a
sequential program. They schedule these threads so as they improve the
program's data locality. The algorithm relies upon hints provided at
the time of thread creation to determine a thread execution order
likely to reduce cache misses.

Work-stealing schedulers are gaining in popularity as scheduling
algorithms for dynamic lightweight task scheduling. Thus, most
research on locality-aware scheduling is done with work-stealing
schedulers nowadays.

\textcite{Acar2000} present a theoretical bound for the number of
cache misses for the work-stealing algorithm and also provide an
implementation of a work-stealing scheduler using locality information
for single-core SMP systems. Our locality-aware work-stealing
scheduler is designed for multi-core SMPs.

\textcite{Guo2010} introduce SLAW, a scalable locality-aware adaptive
work-stealing scheduler. The SLAW scheduler is designed for
programming models where locality hints are provided to the runtime by
the programmer or compiler. Like our locality-aware scheduler, SLAW
also groups workers into places. In contrast to our scheduler, their
implementation disables cross-place steals. SLAW additionally supports
adaptive scheduling and selects a work-first vs. help-first policy for
a task at runtime.

\textcite{Zeldovich2003} present a new asynchronous programming
library which allows event-driven applications to take advantage of
multiprocessors by running code for event handlers in parallel using
work-stealing scheduling. To control the concurrency between events,
the programmer can specify a color for each event: events with the
same color are handled serially; events with different colors can be
handled in parallel. \textcite{Gaud2010} extend the previous work by
introducing heuristics aimed at improving the performance of the
work-stealing algorithm. These heuristics try to preserve cache
locality and avoid unfavorable stealing attempts, with little
involvement required from the application programmers.


%%%% Industry

X10: \textcite{Charles2005} ``X10: an object-oriented approach to
non-uniform cluster computing'', \textcite{Saraswat2010} ``Report on
the programming language X10''

\textcite{Yan2009}: Hierarchical Place Trees: A Portable Abstraction
for Task Parallelism and Data Movement

\textcite{Agarwal2008}: Static Detection of Place Locality and
Elimination of Runtime Checks

Habanero Java

Threading Building Blocks \cite{Contreras2008, Reinders2007} also use
a work-stealing scheduler. The implementation is designed with caches
in mind and tries to limit the unnecessary movement of tasks and
data. When a task has to be passed to a different processor core for
execution, Threading Building Blocks moves the task with the least
likelihood of having data in the cache for the processor core from
which the task is stolen.

%%%% Parallel Depth First Scheduler

\textcite{Blelloch2004}: Effectively sharing a cache among threads

\textcite{Blelloch2008}: Provably good multicore cache performance for
divide-and-conquer algorithms

\textcite{Liaskovitis2006}: Brief announcement: parallel depth first
vs. work stealing schedulers on CMP architectures

\textcite{Chen2007}: Scheduling threads for constructive cache sharing
on CMPs

\textcite{Tam2007}: Thread clustering: sharing-aware scheduling on
SMP-CMP-SMT multiprocessors

\todo{Finish chapter ``Related Work''}


%%% Local Variables: 
%%% mode: latex
%%% TeX-master: "thesis"
%%% End: 