%==============================================================================
% locality-related-work.tex
%==============================================================================

\chapter{Related Work}
\label{chap:locality-related-work}

\begin{itemize}
\item[\textbullet] The data locality of work stealing \cite{Acar2002}
\item[\textbullet] Static Detection of Place Locality and Elimination
  of Runtime Checks \cite{Agarwal2008}
\item[\textbullet] Efficient Optimization of Memory Accesses in
  Parallel Programs \cite{Barik2009}
\item[\textbullet] Scalable Work Stealing \cite{Dinan2009}
\item[\textbullet] Thread scheduling for cache locality \cite{Philbin1996}
\item[\textbullet] Using processor-cache affinity information in
  shared-memory multiprocessor scheduling \cite{Squillante1993}
\item[\textbullet] Thread clustering: sharing-aware scheduling on
  SMP-CMP-SMT multiprocessors \cite{Tam2007}
\item[\textbullet] Hierarchical Place Trees: A Portable Abstraction
  for Task Parallelism and Data Movement \cite{Yan2009}
\item[\textbullet] Mely: Efficient Workstealing for Multicore
  Event-Driven Systems \cite{Gaud2010}
\item[\textbullet] SLAW: a Scalable Locality-aware Adaptive
  Work-stealing Scheduler \cite{Guo2010}
\item[\textbullet] X10: an object-oriented approach to non-uniform
  cluster computing \cite{Charles2005}
\item[\textbullet] Report on the programming language X10
  \cite{Saraswat2010}
\end{itemize}

Threading Building Blocks is designed with caches in mind and works to
limit the unnecessary movement of tasks and data. When a task has to
be passed to a different processor core for execution, Threading
Building Blocks moves the task with the least likelihood of having
data in the cache for the processor core from which the task is
stolen.

\todo{Finish chapter ``Related Work''}

%%% Local Variables: 
%%% mode: latex
%%% TeX-master: "thesis"
%%% End: 