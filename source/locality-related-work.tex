%==============================================================================
% locality-related-work.tex
%==============================================================================

\chapter{Related Work}
\label{chap:locality-related-work}

Locality-aware scheduling is a popular area of research. In the
beginning, most research was done with shared task pools. As
work-stealing scheduling is gaining in popularity, a lot of research
on locality-aware scheduling is done with work-stealing schedulers
nowadays. An alternative algorithm to work-stealing is parallel
depth-first scheduling which is designed specifically for constructive
cache sharing.

\subsubsection{Shared Task Pools}

\textcite{Squillante1993} explore the importance of using
processor-cache affinity information in shared-memory multiprocessor
scheduling. They implement and compare several scheduling algorithms
which trade off load balancing against processor-cache affinity.

\textcite{Philbin1996} use fine-grained threads to decompose a
sequential program. They schedule these threads so as they improve the
program's data locality. Like LASSI, the algorithm relies upon hints
provided at the time of thread creation to determine a thread
execution order likely to reduce cache misses.

\subsubsection{Work-Stealing Scheduler}

\textcite{Acar2000} present a theoretical bound for the number of
cache misses for the work-stealing algorithm. They also provide an
implementation of a locality-aware work-stealing scheduler. This
algorithm is designed for single-core multiprocessor systems, while
LASSI supports chip multiprocessor systems.

X10 \cite{Charles2005, Saraswat2010} is designed for parallel
programming using the partitioned global address space model and place
locality. Computations are divided among a set of places. Each of
those places holds some data and hosts one or more activities that
operate on those data. \textcite{Agarwal2008} present a novel
framework for statically establishing place locality in X10.

The Habanero Java \cite{HJ} language is based on early versions of the
X10 language. Like X10, it supports locality control with task and
data distributions using places. \textcite{Yan2009} introduce
\emph{Hierarchical Place Trees} and integrate them into the Habanero
Java compiler and runtime system. In contrast to our single level
\emph{Work-Stealing Places}, \emph{Hierarchical Place Trees} support
co-location of data and computation at multiple levels of a memory
hierarchy.

\textcite{Guo2010} introduce SLAW, a scalable locality-aware adaptive
work-stealing scheduler. Like LASSI, SLAW groups workers into places
and requires locality hints provided by the programmer or compiler. In
contrast to LASSI, SLAW disables cross-place steals and additionally
supports adaptive scheduling by selecting a work-first or help-first
policy for a task at runtime \cite{Guo2009}.

\textcite{Zeldovich2003} present libasync-smp, an asynchronous
programming library allowing event-driven applications to run code for
event handlers in parallel using work-stealing
scheduling. Event-coloring is used to control the parallelism between
events: events with the same color are handled serially and events
with different colors are handled concurrently. \textcite{Gaud2010}
extend the previous work by introducing heuristics aimed at improving
the performance of the work-stealing algorithm. Like LASSI, the
\emph{locality-aware stealing} heuristics aims to preserve cache
locality. Other heuristics introduced are \emph{time-left stealing}
and \emph{penalty-aware stealing}. Unlike in our locality-aware
scheduler, those heuristics only require little involvement from the
application programmers.

The implementation of Threading Building Blocks \cite{Contreras2008,
  Reinders2007} uses a work-stealing scheduler which tries to limit
unneeded migration of tasks and data. Work-stealing places of our
intervals scheduler can be modified to do the same.

\subsubsection{Parallel Depth-First Scheduler}

Parallel depth-first scheduling was introduced by
\textcite{Blelloch1999}. In parallel depth-first scheduling, tasks are
assigned priorities in the same ordering as they would be executed in
a sequential program. This means, tasks that would be executed earlier
are given higher priorities than those that would be executed
later. As a result, parallel depth-first scheduling tends to employ
constructive cache sharing \cite{Liaskovitis2006, Chen2007} because it
co-schedules threads in a way that simulates the sequential execution
order. In work-stealing scheduling cores tend to have disjointed
working sets. However, the concept of places can be used to enable
constructive cache sharing in work-stealing schedulers.


%%% Local Variables: 
%%% mode: latex
%%% TeX-master: "thesis"
%%% End: 