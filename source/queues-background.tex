%==============================================================================
% queues-background.tex
%==============================================================================

\chapter{Background}
\label{chap:queues-background}

\section{Work-Stealing Queues}
\label{sec:queues-background-work-stealing-queues}

In a work-stealing scheduler each worker has a local queue to maintain
its own pool of ready tasks from which it obtains work. When a worker
finds that its pool is empty, it becomes a thief and steals a task
from the pool of a victim worker chosen at random.

Depending on the desired extraction strategy we can implement the
work-stealing queues differently. Most work-stealing schedulers use
work-stealing deques \cite{Arora1998, Acar2000, Blumofe1995,
  Frigo1998, Danaher2005} but there exist also implementations for
LIFO or FIFO extraction \cite{Michael2009}.

A work-stealing deque is like a traditional deque \cite{Knuth1997}
except that only the deque's owner thread puts and takes local work to
and from the deque's bottom end. Elements are stolen from the top end
of the deque. This minimizes synchronization overhead for the deque's
owner.

All work-stealing queues provide the following three methods in their
interface:

\begin{itemize}
\item \lstinline!put(WorkItem workItem)!: Puts \lstinline!workItem!
  into the queue.
\item \lstinline!WorkItem take()!: Takes an object from the queue and
  returns it if the queue is not empty, otherwise returns
  \lstinline!null!.
\item \lstinline!WorkItem steal()!: Returns \lstinline!null! if the
  queue is empty. Otherwise, returns the element successfully stolen
  from the queue, or \lstinline!null! if this worker loses a race with
  another worker to steal or take a work item.
\end{itemize}

\lstinputlisting[style=Float,
  caption={Work-stealing queue interface}, 
  label=lst:work-stealing-queue-interface]{
    ../listings/queues-background/WorkStealingQueue.java
}

Note that the \lstinline!put()! and \lstinline!take()! methods are
invoked only by the queue's owner.


\section{Current Queue Implementation}
\label{sec:queues-background-current-implementation}

The queue currently used by the intervals scheduler is the
\emph{Work-Stealing Lazy Deque}. This deque is unbounded and uses a
cyclic array to store its work items.

It is called \emph{lazy} because the owner of the deque only lazily
updates the location of the deque's head. This means it only updates
the head when its owner tries to take a work item and finds it was
stolen by a competing thief.

The members of the \emph{Work-Stealing Lazy Deque} are defined as:

\lstinputlisting[style=Skip, nolol,]
%  caption={Work-Stealing Lazy Deque}, 
%  label=lst:work-stealing-lazy-deque]
{
    ../listings/queues-background/WorkStealingLazyDeque.java
}

The \lstinline!workItems! array contains the work items of the
queue. \lstinline!ownerHead! and \lstinline!ownerTail! are indices in
the array and represent the head and tail of the queue for the
owner. \lstinline!thief! represents the head for the thief and is also
the lock object used when a thief tries to steal a work item.

Listing \ref{lst:work-stealing-lazy-deque-put} defines the
\lstinline!put()! method which puts \lstinline!workItem! onto the
bottom of the deque. The method calls \lstinline!expand()! when the
array containing the work items is full (Line
\ref{lst:work-stealing-lazy-deque-put-expand}).

\lstinputlisting[style=FloatNumbers,
  caption={Work-Stealing Lazy Deque: \lstinline!put()! method}, 
  label=lst:work-stealing-lazy-deque-put]{
    ../listings/queues-background/WorkStealingLazyDeque-put.java
}

When the \lstinline!workItems! array is full, \lstinline!put()! calls
the method \lstinline!expand()! (Listing
\ref{lst:work-stealing-lazy-deque-expand}). \lstinline!expand()! can
only be called by the owner of the deque when no thieves are active
(Line \ref{lst:work-stealing-lazy-deque-expand-only-owner}). It
allocates a new doubled size array and copies over the work items from
the old array to the new array. Then it resets the array indices and
replaces the old array with the new one (Lines
\ref{lst:work-stealing-lazy-deque-expand-reset-1} --
\ref{lst:work-stealing-lazy-deque-expand-reset-2}).

\lstinputlisting[style=FloatNumbers,
  caption={Work-Stealing Lazy Deque: \lstinline!expand()! method}, 
  label=lst:work-stealing-lazy-deque-expand]{
    ../listings/queues-background/WorkStealingLazyDeque-expand.java
}

\VerbatimFootnotes Method \lstinline!take()! is defined in Listing
\ref{lst:work-stealing-lazy-deque-take}. It takes an object from the
bottom of the deque if the deque is not empty, otherwise it returns
\lstinline!null!. On Line \ref{lst:work-stealing-lazy-deque-take-cas}
we use Compare-and-Set\footnote{Also known as Compare-and-Swap
  \cite{IBM1974}. \verb!compareAndSet(int i, E expect, E update)!
  atomically sets the element at position \verb!i! to the given object
  \verb!update! if the current value equals \verb!expect!. The method
  returns \verb!true! if successful and \verb!false! else.} to check
if another worker stole the item we wanted to take. If the item is
gone, we know that all previous items must have been stolen too and we
can update our notion of the head of the deque (Line
\ref{lst:work-stealing-lazy-deque-take-update}).

\lstinputlisting[style=SkipNumbers,
  caption={Work-Stealing Lazy Deque: \lstinline!take()! method}, 
  label=lst:work-stealing-lazy-deque-take]{
    ../listings/queues-background/WorkStealingLazyDeque-take.java
}

Listing \ref{lst:work-stealing-lazy-deque-steal} shows the
implementation of \lstinline!steal()! using a synchronized block to
make sure there can only be one thief at any given time. If the deque
is empty, \lstinline!steal()! returns \lstinline!null!. Otherwise, it
returns the element successfully stolen from the top of the deque, or
\lstinline!null! if this process loses a race with another process to
steal the topmost element. \lstinline!null! is also returned if a
\lstinline!steal()! operation lost a race for the last work item
caused by a concurrent \lstinline!take()! operation.

\lstinputlisting[style=FloatNumbers,
  caption={Work-Stealing Lazy Deque: \lstinline!steal()! method}, 
  label=lst:work-stealing-lazy-deque-steal]{
    ../listings/queues-background/WorkStealingLazyDeque-steal.java
}


%%% Local Variables: 
%%% mode: latex
%%% TeX-master: "thesis"
%%% End: 
