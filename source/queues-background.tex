%==============================================================================
% queues-background.tex
%==============================================================================

\chapter{Background}
\label{chap:queues-background}

\section{Work-Stealing Queues}
\label{sec:queues-background-work-stealing-queues}

\lstinputlisting[style=Float,
  caption={Work-stealing queue interface}, 
  label=lst:work-stealing-queue-interface]{
    ../listings/queues-background/WorkStealingQueue.java
}

\subsection{The art of multiprocessor programming \cite{Herlihy2008}}

Ideally, a work-stealing algorithm should provide a linearizable
implementation whose take and steal methods always return a task if
one is available. In practice, however, we can settle for something
weaker, allowing a \lstinline!steal()! call to return \lstinline!null!
if it conflicts with a concurrent \lstinline!steal()! call. Though we
could have the unsuccessful thief simply try again, it makes more
sense in this context to have a thread retry the \lstinline!steal()!
operation on a different, randomly chosen deque each time. To support
such a retry, a \lstinline!steal()! call may return \lstinline!null!
if it conflicts with a concurrent \lstinline!steal()! call.

\todo{Rephrase paragraph}

\subsection{Dynamic circular work-stealing deque \cite{Chase2005}}

The ABP work-stealing algorithm of Arora, Blumofe, and Plaxton
\cite{Arora2001} has been gaining popularity as the multiprocessor
load-balancing technology of choice in both industry and academia
\cite{Arora2001, Acar2002, Blumofe1995, Frigo1998, Danaher2005}. The
scheme implements a provably efficient work-stealing paradigm due to
Blumofe and Leiserson \cite{Blumofe1999} that allows each process to
maintain a local work deque\footnote{Actually, the work-stealing
  algorithm uses a work-stealing deque, which is like a deque
  \cite{Knuth1997} except that only one process can access one end of
  the queue (the ``bottom''), and only pop operations can be invoked
  on the other end (the ``top'').  For brevity, we refer to the data
  structure as a deque in the remainder of the paper.} and steal an
item from others if its deque becomes empty. The deque's owner process
puts and takes local work to and from the deque's bottom end. To
minimize synchronization overhead for the deque's owner, elements are
stolen from the top end of the deque. No elements are added to the top
end of the deque. An ABP deque thus presents three methods in its
interface:

\begin{itemize}
\item \lstinline!put(Object o)!: Puts \lstinline!o! onto the bottom
  of the deque.
\item \lstinline!Object take()!: Takes an object from the bottom of the
  deque if the deque is not empty, otherwise returns
  \lstinline!Empty!.
\item \lstinline!Object steal()!: If the deque is empty, returns
  \lstinline!Empty!. Otherwise, returns the element successfully
  stolen from the top of the deque, or returns \lstinline!Abort! if
  this process loses a race with another process to steal the topmost
  element.\footnote{In our implementation, as we describe, Abort is
    also returned if a steal operation lost a race with an array
    memory reclamation caused by a concurrent \lstinline!take!
    operation.}
\end{itemize}

Note that \lstinline!put! and \lstinline!take! operations
are invoked only by the deque's owner.

\subsection{A dynamic-sized nonblocking work stealing deque
  \cite{Hendler2006, Hendler2006a}}

In work-stealing scheduling, each process tries to work on its newly
created threads locally, and attempts to steal threads from other
processes only when it has no local threads to execute. This way, the
computational overhead of re-balancing is paid by the processes that
would otherwise be idle.

The ABP work-stealing algorithm of Arora, Blumofe, and Plaxton
\cite{Arora2001} has been gaining popularity as the multiprocessor
load-balancing technology of choice in both industry and academia
\cite{Arora2001, Acar2002, Blumofe1995, Frigo1998, Danaher2005}. The
scheme implements a provably efficient work-stealing paradigm due to
Blumofe and Leiserson \cite{Blumofe1999} that allows each process to
maintain a local work deque,\footnote{Actually, the work stealing
  algorithm uses a work stealing deque, which is like a deque
  \cite{Knuth1997} except that only one process can access one end of
  the queue (the ``bottom''), and only pop operations can be invoked
  on the other end (the ``top''). For brevity, we refer to the data
  structure as a deque in the remainder of the paper.} and steal an
item from others if its deque becomes empty. It has been extended in
various ways such as stealing multiple items \cite{Hendler2002} and
stealing in a locality-guided way \cite{Acar2002}. At the core of the
ABP algorithm is an efficient scheme for stealing an item in a
non-blocking manner from an array-based deque, minimizing the need for
costly Compare-and-Swap (CAS)\footnote{The CAS (location, old-value,
  new-value) operation atomically reads a value from location, and
  writes new-value in location if and only if the value read is
  old-value. The operation returns a boolean indicating whether it
  succeeded in updating the location.} synchronization operations when
fetching items locally.

\subsection{A Java fork/join framework \cite{Lea2000}}

To enable efficient and scalable execution, task management must be
made as fast as possible. Creating, putting, and later taking (or,
much less frequently, stealing) tasks are analogs of procedure call
overhead in sequential programs. Lower overhead enables programmers to
adopt smaller task granularities, and in turn better exploit
parallelism.

Task allocation itself is the responsibility of the JVM. Java garbage
collection relieves us of needing to create a special-purpose memory
allocator to maintain tasks. This substantially reduces the complexity
and lines of code needed to implement intervals compared to similar
frameworks in other languages. The basic structure of the deque
employs the common scheme of using a single (although resizable) array
per deque, along with two indices: The top index acts just like an
array-based stack pointer, changing upon put and take. The bottom
index is modified only by steal.

Because the deque array is accessed by multiple threads, sometimes
without full synchronization, yet individual Java array elements
cannot be declared as volatile, each array element is actually a fixed
reference to a little forwarding object maintaining a single volatile
reference. This decision was made originally to ensure conformance
with Java memory rules, but the level of indirection that it entails
turns out to improve performance on tested platforms, presumably by
reducing cache contention due to accesses of nearby elements, which
are spread out a bit more in memory due to the indirection.

The main challenges in deque implementation surround synchronization
and its avoidance. Even on JVMs with optimized synchronization
facilities, the need to obtain locks for every put and take operation
becomes a bottleneck.  However, adaptations of tactics taken in Cilk
\cite{Frigo1998} provide a solution based on the following
observations:

\begin{itemize}
\item The put and take operations are only invoked by owner threads.
\item Access to the steal operation can easily be confined to one
  stealing thread at a time via an entry lock on steal. This deque
  lock can also serve to disable steal operations when
  necessary. Thus, interference control is reduced to a two-party
  synchronization problem.
\item The take and steal operations can only interfere if the deque is
  about to become empty. Otherwise they are guaranteed to operate on
  disjoint elements of the array.
\end{itemize}

Defining the top and bottom indices as volatile ensures that a take
and steal can proceed without locking if the deque is sure to have
more than one element. This is done via a Dekker-like algorithm in
which put pre-decrements top:

\begin{lstlisting}
if (--top >= bottom) ...
\end{lstlisting}

and steal pre-increments bottom:

\begin{lstlisting}
if (++bottom < top) ...
\end{lstlisting}

In each case they must then check to see if this could have caused the
deque to become empty by comparing the two indices. An asymmetric rule
is used upon potential conflict: take rechecks state and tries to
continue after obtaining the deque lock (the same one as held by
steal), backing off only if the deque is truly empty. A steal
operation instead just backs off immediately, typically then trying to
steal from a different victim. This asymmetry represents the only
significant departure from the otherwise similar THE protocol used in
Cilk.

The use of volatile indices also enables the put operation to proceed
without synchronization unless the deque array is about to overflow,
in which case it must first obtain the deque lock to resize the
array. Otherwise, simply ensuring that top is updated only after the
deque array slot is filled in suppresses interference by any steal.


\section{Current Queue Implementation}
\label{sec:queues-background-current-implementation}

The queue currently used by the intervals scheduler is the lazy
work-stealing deque. It is called ``lazy'' because the owner of the
deque only lazily updates the location of the deque's head. This means
it only updates the head when it tries to take something and finds it
is gone.

\todo{Finish section ``LazyDeque''}

The members of the queue are defined as:

\lstinputlisting[label=lst:work-stealing-lazy-deque]{
    ../listings/queues-background/WorkStealingLazyDeque.java
}

The \lstinline!tasks! array contains the work items of the queue. The
\lstinline!ownerHead! and \lstinline!ownerTail! are indices in the
array and represent the head and tail of the queue for the
owner. \lstinline!thief! is both the lock object used when a thief
tries to steal a work item and also represents the head for the thief.

\lstinputlisting[style=FloatNumbers,
  caption={Lazy deque: Put}, 
  label=lst:work-stealing-lazy-deque-put]{
    ../listings/queues-background/WorkStealingLazyDeque-put.java
}

\lstinputlisting[style=FloatNumbers,
  caption={Lazy deque: Take}, 
  label=lst:work-stealing-lazy-deque-take]{
    ../listings/queues-background/WorkStealingLazyDeque-take.java
}

\lstinputlisting[style=FloatNumbers,
  caption={Lazy deque: Steal}, 
  label=lst:work-stealing-lazy-deque-steal]{
    ../listings/queues-background/WorkStealingLazyDeque-steal.java
}

\lstinputlisting[style=FloatNumbers,
  caption={Lazy deque: Expand}, 
  label=lst:work-stealing-lazy-deque-expand]{
    ../listings/queues-background/WorkStealingLazyDeque-expand.java
}

% \begin{center}
%   \begin{tikzpicture}
%     % \node[text centered,text width=4cm]{Description};
    
%     \begin{scope}[line width=4mm,rotate=270]
%       \newcount\mycount
%       \foreach \angle in {0,360,...,3599}
%       {
%         \mycount=\angle\relax
%         \divide\mycount by 10\relax
%         \draw[black!15,thick] (\the\mycount:2.5cm) -- (\the\mycount:4cm);
%       }
      
%       \draw (162:4.2cm) node[above] {0};
%       \draw (126:4.2cm) node[right] {1};
%       \draw (90:4.2cm) node[right] (topindex) {2};
%       \draw (54:4.2cm) node[below] {3};
%       \draw (18:4.2cm) node[below] {4};
%       \draw (342:4.2cm) node[below] {5};    
%       \draw (306:4.2cm) node[below] {6};
%       \draw (270:4.2cm) node[left] {7};
%       \draw (234:4.2cm) node[left] {8};
%       \draw (198:4.2cm) node[above] {9};
%     \end{scope}  
    
%     \filldraw[gray] (3.25cm,0cm) circle (0.4cm);
    
%     \draw (7cm,0cm) node (topbox) {top};
%     \draw[->] (topbox) -- (topindex);
    
%     \draw[gray] (0,0) circle (4cm) circle (2.5cm);
%   \end{tikzpicture}
% \end{center}


%%% Local Variables: 
%%% mode: latex
%%% TeX-master: "thesis"
%%% End: 
