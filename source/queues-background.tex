%==============================================================================
% queues-background.tex
%==============================================================================

\chapter{Background}
\label{chap:queues-background}

\section{Work-Stealing Queues}
\label{sec:queues-background-work-stealing-queues}

A work-stealing scheduler employs a fixed number of threads called
workers. Each worker has a local queue to maintain its own pool of
ready tasks from which it obtains work. When a worker finds that its
pool is empty, it becomes a thief and steals a task from the pool of a
victim worker chosen at random.

Depending on the desired extraction strategy we can implement the
work-stealing queues differently. Most work-stealing schedulers use
work-stealing deques \cite{Arora2001, Acar2002, Blumofe1995,
  Frigo1998, Danaher2005} but there are also implementations for LIFO
or FIFO extraction \cite{Michael2009}.

A work-stealing deque is like a deque \cite{Knuth1997} except that
only the deque's owner thread puts and takes local work to and from
the deque's bottom end. To minimize synchronization overhead for the
deque's owner, elements are stolen from the top end of the deque.

All work-stealing queues provide the following three methods in their
interface:

\begin{itemize}
\item \lstinline!put(WorkItem workItem)!: Puts \lstinline!workItem!
  into the queue.
\item \lstinline!WorkItem take()!: Takes an object from the queue if
  the queue is not empty, otherwise returns \lstinline!null!.
\item \lstinline!WorkItem steal()!: If the queue is empty, returns
  \lstinline!null!. Otherwise, returns the element successfully stolen
  from the queue, or returns \lstinline!null! if this worker loses a
  race with another worker to steal or take a work item.
\end{itemize}

\lstinputlisting[style=Float,
  caption={Work-stealing queue interface}, 
  label=lst:work-stealing-queue-interface]{
    ../listings/queues-background/WorkStealingQueue.java
}

Note that the \lstinline!put()! and \lstinline!take()! methods are
invoked only by the queue's owner.


\section{Current Queue Implementation}
\label{sec:queues-background-current-implementation}

\todo[inline]{Finish section ``Current Queue Implementation''}

The queue currently used by the intervals scheduler is the lazy
work-stealing deque. It is called ``lazy'' because the owner of the
deque only lazily updates the location of the deque's head. This means
it only updates the head when it tries to take something and finds it
is gone.

The members of the lazy deque are defined as:

\lstinputlisting[style=Listing,
  caption={Lazy deque}, 
  label=lst:work-stealing-lazy-deque]{
    ../listings/queues-background/WorkStealingLazyDeque.java
}

The \lstinline!workItems! array contains the work items of the
queue. The \lstinline!ownerHead! and \lstinline!ownerTail! are indices
in the array and represent the head and tail of the queue for the
owner. \lstinline!thief! represents the head for the thief and is also
the lock object used when a thief tries to steal a work item.

Listing \ref{lst:work-stealing-lazy-deque-put} defines the
\lstinline!put()! method which puts \lstinline!workItem! onto the
bottom of the deque. The method automatically expands the array
containing the work items if it is full and fixes the indices in case
they are about to roll over (Lines
\ref{lst:work-stealing-lazy-deque-put-expand-1} --
\ref{lst:work-stealing-lazy-deque-put-expand-2}).

\lstinputlisting[style=FloatNumbers,
  caption={Lazy deque: \lstinline!put()! method}, 
  label=lst:work-stealing-lazy-deque-put]{
    ../listings/queues-background/WorkStealingLazyDeque-put.java
}

Method \lstinline!take()! is defined in Listing
\ref{lst:work-stealing-lazy-deque-take}. \lstinline!take()! returns
either \lstinline!null! or a work item that should be executed.

At the core of the ABP algorithm is an efficient scheme for stealing
an item in a non-blocking manner from an array-based deque, minimizing
the need for costly Compare-and-Swap (CAS)\footnote{The CAS (location,
  old-value, new-value) operation atomically reads a value from
  location, and writes new-value in location if and only if the value
  read is old-value. The operation returns a boolean indicating
  whether it succeeded in updating the location.} synchronization
operations when fetching items locally.

\lstinputlisting[style=FloatNumbers,
  caption={Lazy deque: \lstinline!take()! method}, 
  label=lst:work-stealing-lazy-deque-take]{
    ../listings/queues-background/WorkStealingLazyDeque-take.java
}

\begin{itemize}
\item \lstinline!WorkItem take()!: Takes an object from the bottom of
  the deque if the deque is not empty, otherwise returns
  \lstinline!null!.
\item \lstinline!WorkItem steal()!: If the deque is empty, returns
  \lstinline!null!. Otherwise, returns the element successfully stolen
  from the top of the deque, or returns \lstinline!null! if this
  process loses a race with another process to steal the topmost
  element. \lstinline!null! is also returned if a \lstinline!steal!
  operation lost a race with an array memory reclamation caused by a
  concurrent \lstinline!take! operation.
\end{itemize}


Listing \ref{lst:work-stealing-lazy-deque-steal}

\lstinputlisting[style=FloatNumbers,
  caption={Lazy deque: \lstinline!steal()! method}, 
  label=lst:work-stealing-lazy-deque-steal]{
    ../listings/queues-background/WorkStealingLazyDeque-steal.java
}

Listing \ref{lst:work-stealing-lazy-deque-expand}

\lstinputlisting[style=FloatNumbers,
  caption={Lazy deque: \lstinline!expand()! method}, 
  label=lst:work-stealing-lazy-deque-expand]{
    ../listings/queues-background/WorkStealingLazyDeque-expand.java
}

\subsection{A Java fork/join framework \cite{Lea2000}}

Task allocation itself is the responsibility of the JVM. Java garbage
collection relieves us of needing to create a special-purpose memory
allocator to maintain tasks. This substantially reduces the complexity
and lines of code needed to implement intervals compared to similar
frameworks in other languages. The basic structure of the deque
employs the common scheme of using a single (although resizable) array
per deque, along with two indices: The top index acts just like an
array-based stack pointer, changing upon put and take. The bottom
index is modified only by steal.

Because the deque array is accessed by multiple threads, sometimes
without full synchronization, yet individual Java array elements
cannot be declared as volatile, each array element is actually a fixed
reference to a little forwarding object maintaining a single volatile
reference. This decision was made originally to ensure conformance
with Java memory rules, but the level of indirection that it entails
turns out to improve performance on tested platforms, presumably by
reducing cache contention due to accesses of nearby elements, which
are spread out a bit more in memory due to the indirection.

The main challenges in deque implementation surround synchronization
and its avoidance. Even on JVMs with optimized synchronization
facilities, the need to obtain locks for every put and take operation
becomes a bottleneck.  However, adaptations of tactics taken in Cilk
\cite{Frigo1998} provide a solution based on the following
observations:

\begin{itemize}
\item The put and take operations are only invoked by owner threads.
\item Access to the steal operation can easily be confined to one
  stealing thread at a time via an entry lock on steal. This deque
  lock can also serve to disable steal operations when
  necessary. Thus, interference control is reduced to a two-party
  synchronization problem.
\item The take and steal operations can only interfere if the deque is
  about to become empty. Otherwise they are guaranteed to operate on
  disjoint elements of the array.
\end{itemize}

Defining the top and bottom indices as volatile ensures that a take
and steal can proceed without locking if the deque is sure to have
more than one element. This is done via a Dekker-like algorithm in
which put pre-decrements top:

\begin{lstlisting}
if (--top >= bottom) ...
\end{lstlisting}

and steal pre-increments bottom:

\begin{lstlisting}
if (++bottom < top) ...
\end{lstlisting}

In each case they must then check to see if this could have caused the
deque to become empty by comparing the two indices. An asymmetric rule
is used upon potential conflict: take rechecks state and tries to
continue after obtaining the deque lock (the same one as held by
steal), backing off only if the deque is truly empty. A steal
operation instead just backs off immediately, typically then trying to
steal from a different victim. This asymmetry represents the only
significant departure from the otherwise similar THE protocol used in
Cilk.

The use of volatile indices also enables the put operation to proceed
without synchronization unless the deque array is about to overflow,
in which case it must first obtain the deque lock to resize the
array. Otherwise, simply ensuring that top is updated only after the
deque array slot is filled in suppresses interference by any steal.

% \begin{center}
%   \begin{tikzpicture}
%     % \node[text centered,text width=4cm]{Description};
    
%     \begin{scope}[line width=4mm,rotate=270]
%       \newcount\mycount
%       \foreach \angle in {0,360,...,3599}
%       {
%         \mycount=\angle\relax
%         \divide\mycount by 10\relax
%         \draw[black!15,thick] (\the\mycount:2.5cm) -- (\the\mycount:4cm);
%       }
      
%       \draw (162:4.2cm) node[above] {0};
%       \draw (126:4.2cm) node[right] {1};
%       \draw (90:4.2cm) node[right] (topindex) {2};
%       \draw (54:4.2cm) node[below] {3};
%       \draw (18:4.2cm) node[below] {4};
%       \draw (342:4.2cm) node[below] {5};    
%       \draw (306:4.2cm) node[below] {6};
%       \draw (270:4.2cm) node[left] {7};
%       \draw (234:4.2cm) node[left] {8};
%       \draw (198:4.2cm) node[above] {9};
%     \end{scope}  
    
%     \filldraw[gray] (3.25cm,0cm) circle (0.4cm);
    
%     \draw (7cm,0cm) node (topbox) {top};
%     \draw[->] (topbox) -- (topindex);
    
%     \draw[gray] (0,0) circle (4cm) circle (2.5cm);
%   \end{tikzpicture}
% \end{center}


%%% Local Variables: 
%%% mode: latex
%%% TeX-master: "thesis"
%%% End: 
