%==============================================================================
% locality-conclusions.tex
%==============================================================================

\chapter{Conclusions and Future Work}
\label{chap:locality-conclusions}

\section{Conclusions}
\label{sec:locality-conclusions-and-future-work-conclusions}

\todo{Finish section ``Conclusions''}


\section{Future Work}
\label{sec:locality-conclusions-and-future-work-future-work}

Scheduling of lightweight threads is a very broad area of research and
there are many possible directions we could further extend our work.

\subsubsection{API, online contention awareness, more benchmarks}

While we could show that locality-aware work-stealing is beneficial
for the runtime of a certain group of applications, we should write
more benchmarks and come up with heuristics and online feedback to
automate locality placements.

Work-stealing places have to be manually configured for each
system. It would be good to automate this task.

\textcite{Mars2010} Contention Aware Execution: Online Contention
Detection and Response

\textcite{Gaud2010} extend the previous work by introducing heuristics
aimed at improving the performance of the work-stealing
algorithm. Like the locality-aware intervals scheduler, the
\emph{locality-aware stealing} heuristics aims to preserve cache
locality. Other heuristics introduced are \emph{time-left stealing}
and \emph{penalty-aware stealing}. Unlike in our locality-aware
scheduler, those heuristics only require little involvement from the
application programmers.

\subsubsection{Data Affinity}

\textcite{Charles2005, Saraswat2010}: X10

\textcite{Guo2010} SLAW: a Scalable Locality-aware Adaptive
Work-stealing Scheduler

\subsubsection{Parallel Depth First Scheduler}

\textcite{Liaskovitis2006}: Brief announcement: parallel depth first
vs. work stealing schedulers on CMP architectures

\textcite{Chen2007} Scheduling threads for constructive cache sharing
on CMPs

\textcite{Fatourou2001} Low-contention depth-first scheduling of
parallel computations with write-once synchronization variables

Parallel depth-first scheduling was introduced by
\textcite{Blelloch1999}. In parallel depth-first scheduling, tasks are
assigned priorities in the same ordering as they would be executed in
a sequential program. This means, tasks that would be executed earlier
are given higher priorities than those that would be executed
later. As a result, parallel depth-first scheduling tends to employ
constructive cache sharing \cite{Liaskovitis2006, Chen2007} as it
co-schedules threads in a way that simulates the sequential execution
order. In work-stealing scheduling cores tend to have disjointed
working sets. However, the concept of places can be used to enable
constructive cache sharing in work-stealing schedulers.

\subsubsection{Avoid counter productive steals}

Load balancing across work-stealing places could lead to
counter-productive stealing. One possible direction for future work
would be to avoid counter-productive steals.

\textcite{Gaud2010} extend the previous work by introducing heuristics
aimed at improving the performance of the work-stealing
algorithm. Like the locality-aware intervals scheduler, the
\emph{locality-aware stealing} heuristics aims to preserve cache
locality. Other heuristics introduced are \emph{time-left stealing}
and \emph{penalty-aware stealing}. Unlike in our locality-aware
scheduler, those heuristics only require little involvement from the
application programmers.

\subsubsection{Adaptive number of workers}

The affinity of the workers is set such that they execute on different
cores. While this eliminates interference between the worker threads,
they will nevertheless share their assigned core with other processes
in the system, subject to standard Linux scheduling policy. 

\textcite{Agrawal2006}: An empirical evaluation of work stealing with
parallelism feedback

\textcite{Agrawal2007}: Adaptive work-stealing with parallelism
feedback

\subsubsection{Adaptive scheduling policies}

SLAW additionally supports adaptive scheduling and selects a
work-first vs. help-first policy for a task at runtime.

\textcite{Guo2009} Work-first and help-first scheduling policies for
async-finish task parallelism

\textcite{Guo2010} SLAW: a Scalable Locality-aware Adaptive
Work-stealing Scheduler

\todo{Finish section ``Future Work''}

\todo{Finish chapter ``Conclusions and Future Work''}


%%% Local Variables: 
%%% mode: latex
%%% TeX-master: "thesis"
%%% End: 
