%==============================================================================
% locality-approach.tex
%==============================================================================

\chapter{Approach}
\label{chap:locality-approach}

\todo[inline]{Finish chapter ``Approach''}

\begin{itemize}
\item Implemented locality-aware benchmarks using threads
\item Could show that locality matters so implemented benchmarks with
  intervals
\item Rewrote intervals thread pool supporting locality
\end{itemize}

\section{Locality-Aware Benchmarks}

We evaluate the locality-aware implementation of \emph{intervals} on a
variety of benchmarks. To reduce the impact of JVM overheads in the
evaluation, including JIT compilation and garbage collection, the
execution time reported is the average of the three best benchmark
iterations from three seperate VM incocations. Each VM invocation
performs 10 benchmark iterations.

The JVM used on both machines described in appendix
\ref{chap:experimental-setup} is Sun Hotspot JDK 1.6. In both cases,
the JVM was invoked with the following parameters:

\begin{verbatim}
    -server -Xmx4096M -Xms4096M -Xss8m -XX:+UseNUMA
\end{verbatim}

The following benchmarks were first written to use threads and then
ported over to use \emph{intervals}.

\subsection*{Cache-Stress Test}

\todo[inline]{Describe benchmark ``Cache-Stress Test''}

\subsection*{Merge Sort}

\todo[inline]{Describe benchmark ``Merge Sort''}

\subsection*{Block Matrix Multiplication}

\todo[inline]{Describe benchmark ``Block Matrix Multiplication''}


%%% Local Variables: 
%%% mode: latex
%%% TeX-master: "thesis"
%%% End: 