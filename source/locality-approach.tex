%==============================================================================
% locality-approach.tex
%==============================================================================

\chapter{Approach}
\label{chap:locality-approach}

As a motivating example we implement a synthetic multi-threaded
locality-aware benchmark called \emph{Cache Stress Test}. 

If a multi-threaded benchmark with best possible locality has better
performance and fewer last-level cache misses than the same benchmark
with no specific locality, we should be able to see the same effect
when porting the benchmark to intervals. Hence, we know whether it
makes sense to design a locality-aware intervals scheduler.

\section{Benchmark}

We wrote our benchmark for the Intel Nehalem system with two
processors and eight cores. While every core has its separate level 1
and level 2 caches, the per-processor 8 MB level 3 cache is shared
between all cores of the same processor (Appendix
\ref{sec:experimental-setup-mafushi}).

\emph{Cache Stress Test} first randomly initializes two integer arrays
of size \numprint{2097144}. The size of each array is about 8 MB,
which is the same as the size of the last level cache per
processor. Then we create 64 threads with their affinity set to a
specific processor: 32 are run on the first processor and 32 on the
second processor. One half of the threads operates on the elements of
the first array and the other half operates on the elements of the
second array. Each thread adds and multiplies all the elements of its
respective array 100 times.

We implemented several different variants of the \emph{Cache Stress
  Test}:

\begin{description}
\item[Best Locality:] All the threads working on the first array
  have affinity to the first processor and all threads working on the
  second array have affinity to the second processor.
\item[Ignorant Locality:] The threads are not bound to any
  specific processor, i.e. they are \emph{ignorant} of their locality.
\item[Random Locality:] The affinity of the threads is set to a
  \emph{random} processor.
\item[Worst Locality:] Half the threads with affinity to the first
  processor work on the first array, and the other half works on the
  second array and vice versa.
\end{description}


\section{Performance Evaluation}


This warms up the cache and if all the threads working on the same
array are also running on the same processor, minimizes last level
cache misses. We call this the \emph{best locality}. The \emph{worst
  locality} would be if half the threads with affinity to cores on the
first processor work on the first array and the other half works on
the second array and vice versa. When using \emph{random locality} we
set the affinity of the threads randomly. Threads without any locality
set are said to be of \emph{ignorant} of their locality.

\todo{Finish chapter ``Approach''}

Could show that locality matters so implemented benchmarks with
interval.

Rewrote intervals thread pool supporting locality.


%%% Local Variables: 
%%% mode: latex
%%% TeX-master: "thesis"
%%% End: 