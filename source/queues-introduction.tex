%==============================================================================
% queues-introduction.tex
%==============================================================================

\part{Work-Stealing Queue Implementations}
\label{part:queues}

\chapter{Introduction}
\label{chap:queues-introduction}

\section{Motivation}
\label{sec:queues-intro-motivation}

The intervals implementation uses a work-stealing scheduler. It
employs a fixed number of threads called workers. Each worker has a
local double-ended queue, or deque, to maintain its own pool of ready
intervals from which it obtains work. When a worker finds that its
pool is empty, it becomes a thief and steals an interval from the pool
of a victim worker chosen at random.

In the non-blocking \footnote{Non-blocking -- in contrast to wait-free
  \cite{Herlihy1991} -- means that it is possible for a worker to
  starve while trying to steal from other workers. Live-locks cannot
  occur as if one worker starves, then others must be making
  progress.} work-stealing algorithm, the deques are implemented with
non-blocking synchronization \cite{Arora2001}. That is, instead of
using mutual exclusion, it uses powerful atomic synchronization
primitives such as compare-and-swap \cite{Moir1997}. The current deque
implementation of intervals however uses mutual exclusion when trying
to steal.

Our hypothesis was that we could improve the performance of the
intervals scheduler with non-blocking deques. Thus, as a separate
effort, we designed and explored alternative non-blocking queue
implementations.

\section{Overview}
\label{sec:queues-intro-overview}

\todo{Rewrite ``Overview''}

The goal of this part of the thesis was to find queue implementations
such that they improve the performance of the intervals scheduler.

None of the approaches that were developed as part of this research
yielded a deque that was improving work-stealing performance on the
machines we had to test them with (see appendix
\ref{sec:experimental-setup-marvin} and
\ref{sec:experimental-setup-mafushi}).

\todo{List chapters and give a short description, mention
  \cite{Saha2007}}

This part of the thesis describes the developed deques and encountered
problems in order to preserve this research for future reference.

% Chapter two introduces the different terms involved when talking about
% virtual machines and different techniques to build them. It also
% summarizes some important hardware aspects. Chapter three analyzes
% other implementations of virtual machines done previously. It
% categorizes them and relates them to the techniques introduced in
% chapter two. Chapter four offers some background on Barrelfish,
% especially those parts which are important to understand the
% interfaces used within VMkit. In chapter five the global approach and
% the design is shown. It describes how the virtual machine is laid out
% and what the properties with respect to the background on virtual
% machines are. Chapter six describes the implementation of the whole
% system within Barrelfish. An evaluation of the implementation is
% presented in chapter seven and the work concludes in chapter eight
% where the implementation is rated according to the evaluation and the
% aim.

%%% Local Variables: 
%%% mode: latex
%%% TeX-master: "thesis"
%%% End: 
