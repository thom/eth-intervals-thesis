%==============================================================================
% queues-introduction.tex
%==============================================================================

\part{Work-Stealing Queue Implementations}
\label{part:queues}

\chapter{Introduction}
\label{chap:queues-introduction}

\section{Motivation}
\label{sec:queues-intro-motivation}

The intervals implementation uses a work-stealing scheduler that
employs a fixed number of threads called workers. Each worker has a
local double-ended queue, or deque, to maintain its own pool of ready
intervals from which it obtains work. When a worker's pool becomes
empty, it tries to steal an interval from the pool of a victim worker
chosen at random.

To enable efficient and scalable execution, management of the
intervals must be made as fast as possible. In the non-blocking
work-stealing algorithm,\footnote{Non-blocking -- in contrast to
  wait-free \cite{Herlihy1991} -- means that it is possible for a
  worker to starve while trying to steal from other
  workers. Live-locks cannot occur as if one worker starves, then
  others must be making progress.} the deques are implemented with
non-blocking synchronization \cite{Arora2001}. That is, instead of
using mutual exclusion, it uses atomic synchronization primitives such
as Compare-and-Swap \cite{Moir1997}. The current deque implementation
of intervals however uses mutual exclusion when trying to steal.

Our hypothesis is that we can improve the performance of the intervals
scheduler with non-blocking queues. Thus, as a separate effort, we
design and explore alternative non-blocking queuesx.

\section{Overview}
\label{sec:queues-intro-overview}

Chapter \ref{chap:queues-background} summarizes the properties of
work-stealing queues and introduces the \emph{Work-Stealing Lazy
  Deque}, the queue currently used by the intervals scheduler. Chapter
\ref{chap:queues-implementation} describes the investigated queue
implementations. None of the approaches we developed as part of this
research yielded a queue that was improving work-stealing performance
on the machines we had to test them with (Appendix
\ref{sec:experimental-setup-marvin} and
\ref{sec:experimental-setup-mafushi}). Possible reasons for this are
given in the performance evaluation in Chapter
\ref{chap:queues-performance}. Chapter
\ref{chap:queues-conclusions-and-future-work} concludes and summarizes
our findings and encountered problems in order to preserve this
research for future reference.


%%% Local Variables: 
%%% mode: latex
%%% TeX-master: "thesis"
%%% End: 
