%==============================================================================
% queues-introduction.tex
%==============================================================================

\part{Work-Stealing Queue Implementations}
\label{part:queues}

% Originally, the topic of this master's thesis was to improve the heap
% model for ownership-based verification at the example of Spec\#. The
% goal was to find heap encodings which transfer structural information
% from the static ownership annotations into the abstract model, such
% that ownership related proof obligations would be simpler to solve by
% the verifier.

% None of the approaches that were developed as part of this research
% yielded a heap model that is promising to improve verification
% performance. Therefore, the idea was abandoned and ``Improving Cee''
% became the new main topic of this thesis.

% This part of the thesis describes the developed heap models and
% encountered problems in order to preserve this research for future
% reference.

\chapter{Introduction}
\label{chap:queues-introduction}

The original work-stealing algorithm uses non-blocking algorithms to
implement queue operations \cite{Arora2001}. However, the current
deque implementation of intervals uses a lock when trying to steal. In
\autoref{part:queues} of this thesis we explore alternative
non-blocking queue implementations and compare them to the current
one.

\section{Problem and Motivation}
\label{sec:queues-intro-problem-and-motivation}

\todo[inline]{Finish section ``Problem and Motivation''}

\section{Aim}
\label{sec:queues-intro-aim}

\todo[inline]{Finish section ``Aim''}

\section{Overview}
\label{sec:queues-intro-overview}

\todo[inline]{Finish section ``Overview''}

The JVM used to benchmark the implementation is Sun Hotspot JDK 1.6
(see Appendix \ref{chap:experimental-setup} for information about the
experimental setup).


%%% Local Variables: 
%%% mode: latex
%%% TeX-master: "thesis"
%%% End: 
