%==============================================================================
% benchmarks.tex
%==============================================================================

\chapter{Java Grande Forum Benchmarks}
\label{chap:benchmarks}

We evaluate the intervals scheduler implementation with a variety of
parallel Java Grande Forum benchmarks \cite{Smith2001, Mathew1999,
  Gregg2003} that have been ported to use intervals.

The following is a summary of the benchmarks described in
\cite{Smith2001, Mathew1999, Gregg2003}.

\subsection*{Crypt}

\emph{Crypt} performs IDEA encryption and decryption of an array of
$N$ bytes. This algorithm involves two principle loops, whose
iterations are independent and are divided using fork-join sections
between intervals in a block fashion.

\subsection*{LUFact}

This benchmark solves an $N \times N$ linear system using LU
factorization followed by a triangular solve. Iterations of the double
loop over the trailing block of the matrix are independent and the
work is divided between intervals in a cyclic fashion using fork-join
sections.

\subsection*{SOR}

The benchmark performs iterations of successive over-relaxations on a
$N \times N$ grid. It features an outer loop over iterations and two
inner loops, each looping over the grid. To parallelize the loop over
array rows, a ``red-black'' ordering mechanism is used. The work is
distributed between intervals in a block manner with help of
point-to-point synchronization.

\subsection*{Series}

This benchmark computes the first $N$ fourier coefficients of the
function $f(x) = (x+1)^x$ on the interval $[0,2]$. It uses fork-join
sections to distribute the loop over the Fourier coefficients between
intervals.

\subsection*{MolDyn}

The \emph{MolDyn} benchmark models particles interacting under a
Lennard-Jones potential in a cubic spatial volume with periodic
boundary conditions. The calculation is distributed between intervals
in a cyclic manner and synchronization is done using barriers.

\subsection*{MonteCarlo}

This benchmark is a financial simulation, using Monte Carlo techniques
to price products derived from the price of an underlying asset. The
work is divided between intervals by using fork-join sections.

\subsection*{RayTracer}

The \emph{RayTracer} benchmark measures the performance of a 3D ray
tracer rendering a scene containing 64 spheres at a resolution of $N
\times N$ pixels. The loop over rows of pixels is distributed to
intervals using fork-join sections.


%%% Local Variables: 
%%% mode: latex
%%% TeX-master: "thesis"
%%% End: 