%==============================================================================
% queues-description.tex
%==============================================================================

\chapter{Work-Stealing Queues}
\label{chap:queues-description}

\subsubsection{The art of multiprocessor programming
  \cite{Herlihy2008}}

Ideally, a work-stealing algorithm should provide a linearizable
implementation whose pop methods always return a task if one is
available. In practice, however, we can settle fro something weaker,
allowing a \lstinline!popTop()! call to return \lstinline!null! if it
conflicts with a concurrent \lstinline!popTop()! call. Though we could
have the unsuccessful thief simply try again, it makes more sense in
this context to have a thread retry the \lstinline!popTop()! operation
on a different, randomly chosen deque each time. To support such a
retry, a \lstinline!popTop()! call may return \lstinline!null! if it
conflicts with a concurrent \lstinline!popTop()! call.

\todo{Rephrase paragraph}

\subsubsection{Dynamic circular work-stealing deque \cite{Chase2005}}

The ABP work-stealing algorithm of Arora, Blumofe, and Plaxton
\cite{Arora2001} has been gaining popularity as the multiprocessor
load-balancing technology of choice in both industry and academia
\cite{Arora2001, Acar2002, Blumofe1995, Frigo1998, Danaher2005}. The
scheme implements a provably efficient work-stealing paradigm due to
Blumofe and Leiserson \cite{Blumofe1999} that allows each process to
maintain a local work deque \footnote{Actually, the work-stealing
  algorithm uses a work-stealing deque, which is like a deque
  \cite{Knuth1997} except that only one process can access one end of
  the queue (the ``bottom''), and only Pop operations can be invoked
  on the other end (the ``top'').  For brevity, we refer to the data
  structure as a deque in the remainder of the paper.} and steal an
item from others if its deque becomes empty.  The deque’s owner
process pushes and pops local work to and from the deque's bottom
end. To minimize synchronization overhead for the deque's owner,
stolen elements are taken from the top end of the deque. No elements
are added to the top end of the deque. An ABP deque thus presents
three methods in its interface:

\begin{itemize}
\item \lstinline!pushBottom(Object o)!: Pushes \lstinline!o! onto the
  bottom of the deque.
\item \lstinline!Object popBottom()!: Pops an object from the bottom
  of the deque if the deque is not empty, otherwise returns
  \lstinline!Empty!.
\item \lstinline!Object steal()!: If the deque is empty, returns
  \lstinline!Empty!. Otherwise, returns the element successfully
  stolen from the top of the deque, or returns \lstinline!Abort! if
  this process loses a race with another process to steal the topmost
  element \footnote{In our implementation, as we describe, Abort is
    also returned if a steal operation lost a race with an array
    memory reclamation caused by a concurrent popBottom operation.}.
\end{itemize}

Note that \lstinline!pushBottom! and \lstinline!popBottom! operations
are invoked only by the deque's owner.

Unfortunately, the use of fixed size arrays introduces an inefficient
memory-size/robustness tradeoff: for $n$ processes and total allocated
memory size $m$, one can tolerate at most $\frac{m}{n}$ items in a
deque. Using cyclic arrays, or the reset-on- empty heuristic presented
in the original ABP algorithm \footnote{The reset-on-empty heuristic
  resets top and bottom to point to the beginning of the array
  whenever the deque becomes empty. It was used by the original ABP
  algorithm to make overflow scenarios less frequent.}, reduces the
chance of overflow but does not eliminate it.

\subsubsection{A Java fork/join framework \cite{Lea2000}}

\minisec{Deques}

To enable efficient and scalable execution, task management must be
made as fast as possible. Creating, pushing, and later popping (or,
much less frequently, taking) tasks are analogs of procedure call
overhead in sequential programs. Lower overhead enables programmers to
adopt smaller task granularities, and in turn better exploit
parallelism.

Task allocation itself is the responsibility of the JVM. Java garbage
collection relieves us of needing to create a special-purpose memory
allocator to maintain tasks. This substantially reduces the complexity
and lines of code needed to implement intervals compared to similar
frameworks in other languages. The basic structure of the deque
employs the common scheme of using a single (although resizable) array
per deque, along with two indices: The top index acts just like an
array-based stack pointer, changing upon push and pop. The base index
is modified only by take.

Because the deque array is accessed by multiple threads, sometimes
without full synchronization, yet individual Java array elements
cannot be declared as volatile, each array element is actually a fixed
reference to a little forwarding object maintaining a single volatile
reference. This decision was made originally to ensure conformance
with Java memory rules, but the level of indirection that it entails
turns out to improve performance on tested platforms, presumably by
reducing cache contention due to accesses of nearby elements, which
are spread out a bit more in memory due to the indirection.

The main challenges in deque implementation surround synchronization
and its avoidance. Even on JVMs with optimized synchronization
facilities, the need to obtain locks for every push and pop operation
becomes a bottleneck.  However, adaptations of tactics taken in Cilk
\cite{Frigo1998} provide a solution based on the following
observations:

\begin{itemize}
\item The push and pop operations are only invoked by owner threads.
\item Access to the take operation can easily be confined to one
  stealing thread at a time via an entry lock on take. This deque lock
  can also serve to disable take operations when necessary. Thus,
  interference control is reduced to a two-party synchronization
  problem.
\item The pop and take operations can only interfere if the deque is
  about to become empty. Otherwise they are guaranteed to operate on
  disjoint elements of the array.
\end{itemize}

Defining the top and base indices as volatile ensures that a pop and
take can proceed without locking if the deque is sure to have more
than one element. This is done via a Dekker-like algorithm in which
push pre-decrements top:

\begin{lstlisting}
if (--top >= base) ...
\end{lstlisting}

and take pre-increments base:

\begin{lstlisting}
if (++base < top) ...
\end{lstlisting}

In each case they must then check to see if this could have caused the
deque to become empty by comparing the two indices. An asymmetric rule
is used upon potential conflict: pop rechecks state and tries to
continue after obtaining the deque lock (the same one as held by
take), backing off only if the deque is truly empty. A take operation
instead just backs off immediately, typically then trying to steal
from a different victim. This asymmetry represents the only
significant departure from the otherwise similar THE protocol used in
Cilk.

The use of volatile indices also enables the push operation to proceed
without synchronization unless the deque array is about to overflow,
in which case it must first obtain the deque lock to resize the
array. Otherwise, simply ensuring that top is updated only after the
deque array slot is filled in suppresses interference by any take.

\minisec{Stealing and Idling}

Worker threads in work-stealing frameworks know nothing about the
synchronization demands of the programs they are running. They simply
generate, push, pop, take, manage the status of, and execute
tasks. The simplicity of this scheme leads to efficient execution when
there is plenty of work for all threads. However, this streamlining
comes at the price of relying on heuristics when there is not enough
work; i.e., during startup of a main task, upon its completion, and
around global full-stop synchronization points employed in some
fork/join algorithms.

The main issue here is what to do when a worker thread has no local
tasks and cannot steal one from any other thread. If the program is
running on a dedicated multiprocessor, then one could make the case
for relying on hard busy-wait spins looping to try to steal
work. However, even here, attempted steals increase contention, which
can slow down even those threads that are not idle due to locking
protocols.  Additionally, in more typical usage contexts of this
framework, the operating system should somehow be convinced to try to
run other unrelated runnable processes or threads.

The tools for achieving this in Java are weak, have no guarantees
\cite{Goetz2006}, but usually appear to be acceptable in practice (as
do similar techniques described for Hood \cite{Blumofe1998}). A thread
that fails to obtain work from any other thread lowers its priority
before attempting additional steals, performs Thread.yield between
attempts, and registers itself as inactive in its thread pool. If all
others become inactive, they all block waiting for additional main
tasks. Otherwise, after a given number of additional spins, threads
enter a sleeping phase, where they sleep (for up to 100ms) rather than
yield between steal attempts. These imposed sleeps can cause
artificial lags in programs that take a long time to split their
tasks. But this appears to be the best general-purpose
compromise. Future versions of the framework may supply additional
control methods so that programmers can override defaults when they
impact performance.

\section{Interface}
\label{sec:queues-description-interface}

\lstinputlisting[style=Float,
  caption={Work-stealing queue interface}, 
  label=lst:work-stealing-queue-interface]{
    ../listings/queues-description/WorkStealingQueue.java
}

\todo{Finish section ``Interface''}

\section{Current Queue Implementation}
\label{sec:queues-description-current-implementation}

This is the deque currently used by the intervals scheduler. It is
called ``Lazy'' because the owner of the deque only lazily updates the
location of the head of deque, i.e., only when it tries to take
something and finds it gone.

\todo{Finish section ``LazyDeque''}

\lstinputlisting[style=Numbers,
  caption={Lazy deque}, 
  label=lst:work-stealing-lazy-deque]{
    ../listings/queues-description/WorkStealingLazyDeque.java
}

% \begin{center}
%   \begin{tikzpicture}
%     % \node[text centered,text width=4cm]{Description};
    
%     \begin{scope}[line width=4mm,rotate=270]
%       \newcount\mycount
%       \foreach \angle in {0,360,...,3599}
%       {
%         \mycount=\angle\relax
%         \divide\mycount by 10\relax
%         \draw[black!15,thick] (\the\mycount:2.5cm) -- (\the\mycount:4cm);
%       }
      
%       \draw (162:4.2cm) node[above] {0};
%       \draw (126:4.2cm) node[right] {1};
%       \draw (90:4.2cm) node[right] (topindex) {2};
%       \draw (54:4.2cm) node[below] {3};
%       \draw (18:4.2cm) node[below] {4};
%       \draw (342:4.2cm) node[below] {5};    
%       \draw (306:4.2cm) node[below] {6};
%       \draw (270:4.2cm) node[left] {7};
%       \draw (234:4.2cm) node[left] {8};
%       \draw (198:4.2cm) node[above] {9};
%     \end{scope}  
    
%     \filldraw[gray] (3.25cm,0cm) circle (0.4cm);
    
%     \draw (7cm,0cm) node (topbox) {top};
%     \draw[->] (topbox) -- (topindex);
    
%     \draw[gray] (0,0) circle (4cm) circle (2.5cm);
%   \end{tikzpicture}
% \end{center}


%%% Local Variables: 
%%% mode: latex
%%% TeX-master: "thesis"
%%% End: 
