%==============================================================================
% queues-conclusions.tex
%==============================================================================

\chapter{Conclusions and Future Work}
\label{chap:queues-conclusions-and-future-work}

\section{Conclusions}
\label{sec:queues-conclusions-and-future-work-conclusions}

Our hypothesis was that we could improve the performance of the
intervals scheduler with non-blocking work-stealing queues. Thus, we
designed and implemented several work-stealing queues.

We evaluated the performance of our queue implementations by using the
intervals implementations of various Java Grande Forum
benchmarks. None of the work-stealing queues we developed
significantly improves work-stealing performance on the machines we
had to test them with: The runtime of the benchmarks using
\emph{Work-Stealing Deque} and \emph{Idempotent Work-Stealing Deque}
compared to the original \emph{Work-Stealing Lazy Deque} are almost
the equal. The \emph{Dynamic Work-Stealing Deque}, \emph{Idempotent
  FIFO Work-Stealing Queue}, and \emph{Idempotent LIFO Work-Stealing
  Queue} are often significantly slower than the other queues. The
\emph{Duplicating Work-Stealing Queue}'s performance is comparable to
the \emph{Idempotent Work-Stealing Deque}.

Apart from the more complex implementation in comparison to the
original \emph{Work-Stealing Lazy Deque}, a possible reason for this
could be the rather small number of cores of our benchmark machines
(Appendix \ref{chap:experimental-setup}). As \textcite{Saha2007}
state, there is no noticeable difference between the speedup of
work-stealing and a global shared work queue when not using more than
8 cores. Our experiments confirmed those findings for the JGF
benchmarks. There is no significant difference between the runtimes of
the different implementations -- whether we use work-stealing or the
single shared work queue, we get almost the same results.


\section{Future Work}
\label{sec:queues-conclusions-and-future-work-future-work}

Given the results of the performance evaluation, a direction for
future research would be to explore the performance of the different
work-stealing queue implementations on machines featuring more than 8
cores.

It may also be interesting to see how our work-stealing queues would
benefit from using the steal-half algorithm of \textcite{Hendler2002}.

One disadvantage of the queues using arrays to maintain work items is
that they do not shrink them again once the queue contains less work
items which might lead to a waste of memory. \textcite{Chase2005}
present a way of shrinking an array without copying by keeping the
reference to the smaller array when expanding. When the queue shrinks
back its array to the previous array, only the elements that were
modified while the larger array was active need to be copied.

Another optimization \textcite{Chase2005} mention is working with a
shared pool of arrays. With the shared pool implementation, whenever
the queue needs a larger array, it allocates one of the appropriate
size from the pool, and whenever it shrinks to a smaller array and
does not need the larger array anymore, it can return it to the pool.


%%% Local Variables: 
%%% mode: latex
%%% TeX-master: "thesis"
%%% End: 
