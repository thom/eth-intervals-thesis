%==============================================================================
% queues-implementation.tex
%==============================================================================

\chapter{Investigated Queues}
\label{chap:queues-implementation}

Besides the \emph{Work-Stealing Deque} (Section
\ref{sec:queues-implementation-ws-deque}) and \emph{Idempotent
  Work-Stealing Deque} (Section
\ref{sec:queues-implementation-idempotent-ws-deque}) we also
implemented the alternative work-stealing queues \emph{Dynamic
  Work-Stealing Deque} (Section
\ref{sec:queues-alternative-implementations-dynamic-deque}) and
\emph{Duplicating Work-Stealing Queue} (Section
\ref{sec:queues-alternative-implementations-duplicating-queue}).

\section{Work-Stealing Deque}
\label{sec:queues-implementation-ws-deque}

The \emph{Work-Stealing Deque} is an unbounded double-ended queue that
dynamically resizes itself as needed. Its design is based on the
\emph{Dynamic Circular Work-Stealing Deque} \cite{Chase2005, Lev2005}.

The \lstinline!WorkStealingDeque! class has three fields,
\lstinline!workItems!, \lstinline!bottom!, and \lstinline!top!:

\lstinputlisting[style=Skip, nolol,]
%  caption={Work-Stealing Deque}, 
%  label=lst:work-stealing-deque,]
{
    ../listings/queues-implementation/WorkStealingDeque.java
}

The array \lstinline!workItems! is used in a cyclic way with
\lstinline!top! and \lstinline!bottom! indicating the two ends of the
deque. An important property of \lstinline!top! is that it is never
decreased. If \lstinline!bottom! is less than or equal to
\lstinline!top!, the deque is empty.

The \lstinline!put()! method (Listing
\ref{lst:work-stealing-deque-put}) first checks whether the current
circular array is full (Line
\ref{lst:work-stealing-deque-put-size}). If it is full, we call
\lstinline!expand()! (Line \ref{lst:work-stealing-deque-put-expand})
to enlarge it by copying the deque's elements into a bigger array. Now
we can put the new work item in the location specified by
\lstinline!bottom!, and then increment \lstinline!bottom! by 1 (Line
\ref{lst:work-stealing-deque-put-update-bottom}).

\lstinputlisting[style=FloatNumbers,
  caption={Work-Stealing Deque: \lstinline!put()! method}, 
  label=lst:work-stealing-deque-put]{
    ../listings/queues-implementation/WorkStealingDeque-put.java
}

Listing \ref{lst:work-stealing-deque-expand} shows the
\lstinline!expand()! method. It allocates a new doubled size array and
copies the old array's elements into the new array. The use of modular
arithmetic ensures that even though the array has changed size and the
work items may have shifted positions, there is no need to update the
\lstinline!top! or \lstinline!bottom! fields.

\lstinputlisting[style=FloatNumbers,
  caption={Work-Stealing Deque: \lstinline!expand()! method}, 
  label=lst:work-stealing-deque-expand]{
    ../listings/queues-implementation/WorkStealingDeque-expand.java
}

In Listing \ref{lst:work-stealing-deque-take} we define the
\lstinline!take()! method. If the deque is empty, we reset it to an
empty state where \lstinline!bottom == top! and return
\lstinline!null! (Lines \ref{lst:work-stealing-deque-take-empty-1} --
\ref{lst:work-stealing-deque-take-empty-2}). If taking a work item
does not make the deque empty, the owner can take it without using a
Compare-and-Swap (CAS) operation (Lines
\ref{lst:work-stealing-deque-take-non-empty-1} --
\ref{lst:work-stealing-deque-take-non-empty-2}). If the owner is
trying to take the last work item, then it must perform a CAS on
\lstinline!top! to see if it won or lost any race with a concurrent
\lstinline!steal()! operation to take the last item (Line
\ref{lst:work-stealing-deque-take-cas}). Regardless whether the CAS
operation succeeds, the value of \lstinline!top! is incremented by 1
and the deque is empty: If the CAS in \lstinline!take()! fails, then
some concurrent \lstinline!steal()!  operation succeeded in stealing
the last work item and incremented \lstinline!top!. Therefore the
operation completes by storing the incremented top value in
\lstinline!bottom! which resets the deque to an empty state (Line
\ref{lst:work-stealing-deque-take-update}).

\lstinputlisting[style=FloatNumbers,
  caption={Work-stealing Deque: \lstinline!take()! method},
  label=lst:work-stealing-deque-take]{
    ../listings/queues-implementation/WorkStealingDeque-take.java
}

The \lstinline!steal()! method (Listing
\ref{lst:work-stealing-deque-steal}) first reads \lstinline!top!, then
\lstinline!bottom!. The order is important: If a thread reads
\lstinline!bottom! after \lstinline!top! and sees it is no greater,
the queue is indeed empty because a concurrent modification of
\lstinline!top! could only have increased the \lstinline!top! value.

If the deque is empty, \lstinline!steal()! returns \lstinline!null!
(Lines \ref{lst:work-stealing-deque-steal-empty-1} --
\ref{lst:work-stealing-deque-steal-empty-2}). Else it reads the
element stored in the \lstinline!top! position of the cyclic array,
and tries to increment \lstinline!top! using a CAS operation. If the
CAS fails, it implies that a concurrent \lstinline!steal()!
successfully removed an element from the deque, so the operation tries
to steal again. Else it returns the element read right before the
successful CAS operation.

To prevent \lstinline!steal()! from returning the deque's last work
item if it was already taken by a concurrent \lstinline!take()! after
\lstinline!bottom! is read (Line
\ref{lst:work-stealing-deque-steal-bottom}), but before the CAS
operation is executed (Line \ref{lst:work-stealing-deque-steal-cas}),
any \lstinline!take()! that empties the deque tries to modify
\lstinline!top! using a CAS operation.

\lstinputlisting[style=SkipNumbers,
  caption={Work-Stealing Deque: \lstinline!steal()! method}, 
  label=lst:work-stealing-deque-steal]{
    ../listings/queues-implementation/WorkStealingDeque-steal.java
}


\section{Idempotent Work-Stealing Deque}
\label{sec:queues-implementation-idempotent-ws-deque}

The \emph{Idempotent Work-Stealing Deque} is based on ideas from
\cite{Leijen2009} and \cite{Michael2009}. It is an unbounded
double-ended queue that can resize itself if needed.

Unlike the \emph{Work-Stealing Lazy Deque} (Section
\ref{sec:queues-background-current-implementation}) or the
\emph{Work-Stealing Deque} (Section
\ref{sec:queues-implementation-ws-deque}), the \emph{Idempotent
  Work-Stealing Deque} does not guarantee that each inserted work item
is extracted \emph{exactly} once. Instead it uses the relaxed
semantics of guaranteeing that each inserted work item is extracted
\emph{at least} once.

While this nondeterminism might be dangerous in many applications, it
is fine for our usage of the deque as we modified the interval's
\lstinline!exec()! method to be idempotent (Section
\ref{sec:queues-implementation-idempotent-ws-deque-interval}).\footnote{If
  the application can tolerate duplicated work, for example parallel
  garbage collectors \cite{Flood2001} or constraint solvers, we do not
  have to make the \verb!exec()! method idempotent.}

\subsection{Idempotent Interval}
\label{sec:queues-implementation-idempotent-ws-deque-interval}

Listing \ref{lst:interval} shows the idempotent interval
implementation. Each interval has an associated state
\lstinline!RunningState! (Line \ref{lst:interval-state}). Upon
initialization, the state is set to \lstinline!INIT! (Line
\ref{lst:interval-init}). The internal \lstinline!exec()!  method
performs an atomic CAS operation to try to switch from
\lstinline!INIT! to \lstinline!RUNNING! (Line
\ref{lst:interval-cas}). If the CAS operation succeeds, the associated
task is executed and the state is set to \lstinline!DONE! afterwards
(Line \ref{lst:interval-done}).

\lstinputlisting[style=SkipNumbers, 
  caption={Idempotent interval},
  label=lst:interval]{ 
    ../listings/queues-implementation/Interval.java 
}

This ensures that each interval is only executed once, or stated
differently: running an interval is an idempotent operation.

Idempotent intervals would also simplify the mailbox style
implementation for locality-aware scheduling \cite{Acar2002}.

\subsection{Implementation}
\label{sec:queues-implementation-idempotent-ws-deque-implementation}

In contrast to the \emph{Work-Stealing Lazy Deque} (Section
\ref{sec:queues-background-current-implementation}) or the
\emph{Work-Stealing Deque} (Section
\ref{sec:queues-implementation-ws-deque}), the \lstinline!take()!
method of the \emph{Idempotent Work-Stealing Deque} does not have to
use an expensive CAS operation.

The \lstinline!IdempotentWorkStealingDeque! class has an inner class,
\lstinline!ArrayData!, and two fields, \lstinline!anchor! and
\lstinline!workItems!:

\lstinputlisting[style=Skip, nolol,]
%  caption={Idempotent Work-Stealing Deque}, 
%  label=lst:idempotent-work-stealing-deque]
{
    ../listings/queues-implementation/IdempotentWorkStealingDeque.java
}

The array \lstinline!workItems! is used in a cyclic way with head and
size encapsulated in an \lstinline!ArrayData! reference. The
\lstinline!ArrayData! reference is maintained by the atomic stamped
reference \lstinline!anchor! together with an integer stamp. Our
algorithm needs to guard against the ABA problem\footnote{If a thread
  reads a value $A$ from a shared variable, computes a new value, and
  then attempts a CAS operation, there is a chance that the CAS
  succeeds even if it should not, because for example between the read
  and the CAS some other thread changes the $A$ back to $B$ and then
  back to $A$ again.} in the \lstinline!steal()! operation and uses
the stamp as an ABA-prevention tag.\footnote{The tag is a specific
  implementation choice. At the abstract level, the algorithm does not
  require the use of the tag mechanism but can use any ABA prevention
  mechanism, like bounded tags \cite{Moir1997} or hazard pointers
  \cite{Michael2004}}

Listing \ref{lst:idempotent-work-stealing-deque-put} shows the
\lstinline!put()! method. First the owner reads the anchor to get the
head and size of the queue as well as the ABA-prevention tag (Lines
\ref{lst:idempotent-work-stealing-deque-put-read-1} --
\ref{lst:idempotent-work-stealing-deque-put-read-2}). Then the owner
checks whether the array is full of not (Line
\ref{lst:idempotent-work-stealing-deque-put-full}). If it is full, the
owner expands the array by calling \lstinline!expand()! and loops
around (Line
\ref{lst:idempotent-work-stealing-deque-put-expand}). Otherwise it
adds the work item at the tail of the queue (Line
\ref{lst:idempotent-work-stealing-deque-put-insert}). In Line
\ref{lst:idempotent-work-stealing-deque-put-update-anchor} the owner
updates the anchor by incrementing the queue's size and ABA-prevention
tag.

\lstinputlisting[style=FloatNumbers,
  caption={Idempotent Work-Stealing Deque: \lstinline!put()! method},
  label=lst:idempotent-work-stealing-deque-put]{
    ../listings/queues-implementation/IdempotentWorkStealingDeque-put.java
}

The method \lstinline!expand()! is defined in Listing
\ref{lst:idempotent-work-stealing-deque-expand}. For the owner to
expand a full queue, it allocates a new array with double the current
capacity (Line
\ref{lst:idempotent-work-stealing-deque-expand-new-array}) and copies
the work items from the current array to the newly allocated one
(Lines \ref{lst:idempotent-work-stealing-deque-expand-copy-1} --
\ref{lst:idempotent-work-stealing-deque-expand-copy-2}). After that,
it sets \lstinline!workItems! to the new array (Line
\ref{lst:idempotent-work-stealing-deque-expand-assign}).

\lstinputlisting[style=FloatNumbers,
  caption={Idempotent Work-Stealing Deque: \lstinline!expand()! method},
  label=lst:idempotent-work-stealing-deque-expand]{
    ../listings/queues-implementation/IdempotentWorkStealingDeque-expand.java
}

Listing \ref{lst:idempotent-work-stealing-deque-take} defines the
method \lstinline!take()!. The owner reads the anchor variable to get
the head and size of the queue, and also the ABA-prevention tag (Lines
\ref{lst:idempotent-work-stealing-deque-take-read-anchor-1} --
\ref{lst:idempotent-work-stealing-deque-take-read-anchor-2}). Then it
checks if the queue is empty (Line
\ref{lst:idempotent-work-stealing-deque-take-check-size}). If it is
empty, \lstinline!take()! returns \lstinline!null!. Else, it reads the
work item at the tail of the queue (Lines
\ref{lst:idempotent-work-stealing-deque-take-get}). In Line
\ref{lst:idempotent-work-stealing-deque-put-update-anchor} the method
updates the anchor by decrementing the queues size to indicate the
extraction of a work item.

\lstinputlisting[style=FloatNumbers,
  caption={Idempotent Work-Stealing Deque: \lstinline!take()! method},
  label=lst:idempotent-work-stealing-deque-take]{
    ../listings/queues-implementation/IdempotentWorkStealingDeque-take.java
}

The \lstinline!steal()! method (Listing
\ref{lst:idempotent-work-stealing-deque-steal}) starts by reading the
anchor variable to get the head and size of the queue as well as the
ABA-prevention tag (Lines
\ref{lst:idempotent-work-stealing-deque-steal-read-anchor-1} --
\ref{lst:idempotent-work-stealing-deque-steal-read-anchor-2}). In Line
\ref{lst:idempotent-work-stealing-deque-steal-check-size} the thread
checks if the queue is empty. If it is empty, \lstinline!steal()!
returns \lstinline!null!. Otherwise it gets a pointer to
\lstinline!workItems! (Line
\ref{lst:idempotent-work-stealing-deque-steal-work-items}) and reads
the work item at the head of the queue (Line
\ref{lst:idempotent-work-stealing-deque-steal-read}).

The \lstinline!compareAndSet()! in Line
\ref{lst:idempotent-work-stealing-steal-cas} checks that no work item
was lost: Checking of the ABA-prevention tag makes sure that since the
reads in Lines
\ref{lst:idempotent-work-stealing-deque-steal-read-anchor-1} --
\ref{lst:idempotent-work-stealing-deque-steal-read-anchor-2} the deque
owner has not overwritten the work item read in Line
\ref{lst:idempotent-work-stealing-deque-steal-read}. If the
\lstinline!compareAndSet()! is successful, it updates the anchor with
the incremented head and decremented size to indicate the stealing and
returns the stolen work item. Else, the thread tries to steal again.

\lstinputlisting[style=FloatNumbers,
  caption={Idempotent Work-Stealing Deque: \lstinline!steal()! method},
  label=lst:idempotent-work-stealing-deque-steal]{
    ../listings/queues-implementation/IdempotentWorkStealingDeque-steal.java
}


\section{Alternative Implementations}
\label{sec:queues-alternative-implementations}

\subsection{Dynamic Work-Stealing Deque}
\label{sec:queues-alternative-implementations-dynamic-deque}

The \emph{Dynamic Work-Stealing Deque} uses a list of small arrays to
manage work items. It is based on the work-stealing deque algorithm
presented by \textcite{Hendler2006}.

The algorithm implements each deque as a doubly linked list of nodes,
each of which is a short array, with \lstinline!top! and
\lstinline!bottom! indicating the two ends of the deque:

\lstinputlisting[style=Skip, nolol,]
%  caption={Dynamic Work-Stealing Deque}, 
%  label=lst:dynamic-work-stealing-deque,]
{
    ../listings/queues-implementation/DynamicWorkStealingDeque.java
}

\lstinline!top! and \lstinline!bottom! are instances of the
\lstinline!Index! class pointing to a deque's node with an offset into
that node's array. For the doubly linked list to have a good
performance, it must only resorts to using a costly CAS operation when
a potential conflict requires it. The potential conflict occurs when a
\lstinline!take()!  and \lstinline!steal()!  concurrently try to
remove the last item from the deque.

To do this, we need to have an efficient mechanism to allow detection
of these boundary situations using the relations between the
\lstinline!top! and \lstinline!bottom! pointers, even though these
point to entries that may reside in different nodes. We observe that
given one pointer -- ignoring which array it resides in -- the
distance of the other, cannot be more than 1 if the deque is
empty. \textcite{Hendler2006} describe the method in more detail.

As it turned out, the dynamic work-stealing deque's complexity -- due
to the extra work required for the list's maintenance -- reflects in
its performance (Section
\ref{sec:queues-performance-alternative-dynamic}).

\subsection{Duplicating Work-Stealing Queue}
\label{sec:queues-alternative-implementations-duplicating-queue}

The \emph{Duplicating Work-Stealing Queue} provides an alternative to
the \emph{Idempotent Work-Stealing Deque}. Its design is based on the
Task Parallel Library (TPL) \cite{Leijen2009}.

Like the \emph{Idempotent Work-Stealing Deque}, the \emph{Duplicating
  Work-Stealing Queue} potentially returns a pushed element more than
once. In particular, the \lstinline!put()!  and \lstinline!take()!
operations behave like normal, but the \lstinline!steal()! operation
is allowed to either take an element and remove it from the queue, or
to just duplicate an element in the queue. By allowing duplication we
can avoid an expensive CAS instruction in the \lstinline!take()!
operation.

As we are also using idempotent intervals together with duplicating
work-stealing queues, this nondeterminism is fine for our usage.

Whereas the idempotent work-stealing deque implementation relies on
atomic CAS operations and uses a tag to prevent the ABA problem, the
duplicating work-stealing queue implementation uses a lock on all but
the critical paths. This simplifies the implementation but we could
not find any drastic performance difference (Section
\ref{sec:performance-alternative-duplicating}).


%%% Local Variables: 
%%% mode: latex
%%% TeX-master: "thesis"
%%% End: 
