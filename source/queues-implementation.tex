%==============================================================================
% queues-implementation.tex
%==============================================================================

\chapter{Investigated Queues}
\label{chap:queues-implementation}

\section{Work-Stealing Deque}
\label{sec:queues-implementation-ws-deque}

\begin{itemize}
\item Dynamic circular work-stealing deque \cite{Chase2005}
\item Nonblocking Cyclic Extendable Deque for the ABP work stealing
  algorithm \cite{Lev2005}
\item The art of multiprocessor programming \cite{Herlihy2008}
\end{itemize}

\todo{Finish section ``Work-Stealing Deque''}

\subsection{Dynamic circular work-stealing deque \cite{Chase2005}}

This paper presents the first work-stealing deque that uses a
dynamic-cyclic-array. The algorithm is remarkably simple, efficient,
and completely eliminates the overflow problem. Also, unlike all
previous algorithms, the \lstinline!top! and \lstinline!bottom! fields
are used merely to indicate the two ends of the deque -- no tag field
is required to eliminate the ABA problem. Therefore, the only
restriction on the deque size is due to integer overflow. A 64-bit
integer is large enough to accommodate 64 years of pushes, pops, and
steals executing at a rate of 4 billion operations per second, so it
appears that this will not be a problem in practice.

\subsubsection{The Basic Algorithm}

\minisec{Overview}

The deque is implemented using a cyclic array, together with two
indexes: \lstinline!top! and \lstinline!bottom!, indicating the two
ends of the deque. Specifically, the \lstinline!bottom! index
indicates the next available slot in the array where the next new
element is pushed, and is incremented on every \lstinline!pushBottom!
operation. The \lstinline!top! index indicates the topmost element in
the deque (if there is any), and is incremented on every
\lstinline!steal! operation. If \lstinline!bottom! is less than or
equal to \lstinline!top!, the deque is empty.

Since the algorithm uses a cyclic array it makes more efficient use of
the array, and does not need the reset-on-empty heuristic used in the
original ABP algorithm. If a \lstinline!pushBottom! operation
discovers that the current circular array is full, it enlarges it by
copying the deque's elements into a bigger array, and pushes the new
element into the new enlarged array. The deque elements stored in the
circular array are indexed modulo its size, and therefore when moving
the elements into a bigger array, there is no need to update
\lstinline!top! or \lstinline!bottom! although the actual array
indexes where the elements are stored might change. Since the only
operation that modifies \lstinline!top! is \lstinline!steal!,
\lstinline!top! is never decremented, and there is no need for a tag
field as in all previous work-stealing algorithms \footnote{We assume
  that \lstinline!top! never overflows. As explained in the previous
  section, with a 64-bit integer implementation this is a very
  reasonable assumption.}.

\minisec{The deque operations}

The pseudocode for the \lstinline!pushBottom!, \lstinline!steal! and
\lstinline!popBottom! operations appears in Figures 1, 2 and 3
respectively. The \lstinline!steal! and \lstinline!popBottom! methods
are similar to their counterparts in the original ABP algorithm. The
main difference is that the new algorithm does not have a tag field in
the \lstinline!top! variable -- instead, it maintains the property
that \lstinline!top! is never decremented. The deque object also has a
private \lstinline!casTop! method (depicted in Figure 1), which
receives two values for \lstinline!top!, old and new, and atomically
modifies \lstinline!top! to the new value if and only if it has the
old value. In practice this method is implemented using the
Compare-And-Swap (CAS) instruction, which is widely available on
current machines.

The \lstinline!pushBottom! operation: As in the original algorithm,
the \lstinline!pushBottom! operation inserts the pushed entry to the
deque simply by writing it in the location specified by
\lstinline!bottom!, and then incrementing \lstinline!bottom! by 1. In
our algorithm, however, the \lstinline!pushBottom! operation is also
responsible for enlarging the array if an element is pushed into an
already-full array. Whether the array is enlarged or not, the abstract
\lstinline!pushBottom! operation takes place when \lstinline!bottom!
is updated at Line 9.

To check whether the current array is full, the operation subtracts
the value of \lstinline!top! from \lstinline!bottom! (which gives the
number of elements that were in the deque when \lstinline!top! was
read), and compares it to the size of the array. For memory
reclamation reasons described in Section 4.1, the
\lstinline!pushBottom! operation always leaves one array cell unused.
If necessary, the \lstinline!pushBottom! operation uses the
\lstinline!grow! method (of the \lstinline!CircularArray! class) to
enlarge the current array.

The growable circular array: The simplest implementation of a growable
circular array is a power-of-two-sized array that grows by doubling
its size. When the array is full, a new doubled size array is
allocated, and the elements are copied from the old array to the new
one. The pseudocode for the \lstinline!CircularArray! class appears in Figure
4. Note that since the elements are indexed modulo the array size, the
actual array indexes where the elements are stored might change when
copying from one array to another, but the value of \lstinline!top!
and \lstinline!bottom! remains the same.

The \lstinline!steal! operation: As in the original algorithm, the
\lstinline!steal! method begins by reading \lstinline!top! and
\lstinline!bottom!, and checking whether the deque is empty by
comparing these values. If the deque is not empty, it reads the
element stored in the \lstinline!top! position of the cyclic array,
and tries to increment \lstinline!top! using a CAS operation. If the
CAS fails, it implies that a concurrent \lstinline!steal! operation
successfully removed an element from the deque, so the operation
returns the Abort value; otherwise it returns the element read right
before the successful CAS operation. Since the algorithm uses a cyclic
array, it is important to read the element from the array before we do
the CAS, because after the CAS completes, this location may be
refilled with a new value by a concurrent \lstinline!pushBottom!
operation.

A successful CAS is the point at which the abstract \lstinline!steal!
operation takes place. Note that because \lstinline!top! is read
before \lstinline!bottom!, it is guaranteed that the values read
represent a consistent view of the memory. Specifically, it implies
that \lstinline!bottom! and \lstinline!top! indeed had their observed
values when \lstinline!bottom! was read at Line 12. A subtle case may
arise, however, if the deque is emptied by a concurrent
\lstinline!popBottom! operation after \lstinline!bottom! is read, but
before the CAS is executed. For that reason, as we describe later, any
\lstinline!popBottom! operation that empties the deque tries to modify
\lstinline!top! (using a CAS operation), to guarantee that no
concurrent \lstinline!steal! operation will also returns the deque's
last entry.

The \lstinline!popBottom! operation: As in the original algorithm, the
owner can pop inexpensively (without using a CAS operation) provided
that doing so does not cause the deque to become empty. If the deque
was already empty, then it simply resets it to a canonical empty state
(\lstinline!bottom == top!) and returns the Empty value
(Lines 26-28). If the deque becomes empty, then the owner must perform
a CAS on \lstinline!top! to see if it won or lost any race (with a
concurrent \lstinline!steal! operation) to pop the last item. Unlike
the original ABP algorithm, the new algorithm performs the CAS on the
value of the \lstinline!top! index and not on a tag value (note that
incrementing \lstinline!top! when the deque is empty leaves the deque
in an empty state). Right after the CAS operation, whether it succeeds
or not, the value of \lstinline!top! is $t + 1$ (note that if the CAS
fails, then some concurrent \lstinline!steal! operation updated
\lstinline!top! to that value). Therefore the deque is empty, and the
operation completes by storing $t + 1$ in \lstinline!bottom! (by that,
resetting the deque to a canonical empty state). In any case that the
\lstinline!popBottom! operation does not return the empty value, the
abstract \lstinline!popBottom! operation takes place when
\lstinline!bottom! is updated at Line 23.

\minisec{Avoid \lstinline!top! accesses in \lstinline!pushBottom!}

Unlike the original ABP algorithm, the new algorithm requires reading
\lstinline!top! on every execution of the \lstinline!pushBottom!
operation. This may result in more data-cache misses compared to the
original algorithm (recall that unlike \lstinline!bottom!,
\lstinline!top! is modified by all processes).

The frequency of accesses to the \lstinline!top! variable can be
significantly reduced, by keeping a local upper bound on the size of
the deque, and only read \lstinline!top! when the upper bound
indicates that an array expansion may be necessary. Such a local upper
bound can be easily achieved by saving the last value of
\lstinline!top! read in a local variable, and using this variable to
compute the size of the deque (instead of the real value of
\lstinline!top!). Because \lstinline!top! is never decremented, the
real size of the deque can only be smaller than the one calculated
using this local variable.

\subsubsection{Shrinking After Growth}

One disadvantage of the algorithm as presented is that it does not
shrink the array as the deque retreats from its maximum. That means
that the memory used by the deque is a constant factor times its
maximum size, which might result in a big waste of memory.

Shrinking the array is no harder than growing the array; it only
requires that the algorithm check against a minimum use fraction of
the current array when performing a \lstinline!popBottom! operation
\footnote{With the current implementation, this fraction must be
  strictly less than $\frac{1}{2}$, to guarantee that the deque
  elements could fit into the smaller array while leaving one array
  cell unused.}. The code for the \lstinline!popBottom! operation with
the possible shrinking operation appears in Figures 5 and 6. As
illustrated by the code, Line 31 was modified to call the
\lstinline!perhapsShrink! method just before returning the popped
value. The \lstinline!perhapsShrink! method shrinks the array if the
number of elements in the deque is less than some fraction
$\frac{1}{K}$ of the array size, where $K \ge 3$. We omit the code for
the \lstinline!CircularArray!'s shrink method since it is almost
identical to the code of this class's \lstinline!grow! method.
Finally, note that the \lstinline!perhapsShrink! method is independent
of the \lstinline!popBottom! operation, and therefore can be invoked
by the deque's owner on other occasions (for example after a
\lstinline!pushBottom! operation).

\minisec{Shrinking without copying}

The simplest way to shrink back to a smaller array is similar to the
way we grow it: Allocate a new smaller array, and copy the data from
the big array to the smaller one. We can save the allocation time,
however, if whenever we extend an array, we retain a reference from
the bigger array to the smaller one. If each array has a reference to
the smaller array from which it was extended, then the garbage
collector cannot deallocate all the arrays that precede the current
active one, and the algorithm can reuse these arrays when shrinking.

Keeping the references to the smaller arrays not only saves the
allocation time, it can also save some of the copying work: when the
algorithm shrinks back from the big array to its previous smaller
array, only the elements that were modified while the bigger array was
active need to be copied (because the smaller array was not
deallocated and therefore was not modified while the bigger array was
active). This can be accomplished by maintaining a low-water-mark with
each array: an integer that indicates the lowest value of
\lstinline!bottom! in which an element was stored while the array was
active. When a deque shrinks its array, only the elements stored in
indexes greater than or equal to the low water mark of the bigger
array are copied. Also, the smaller array's low water mark is updated
to the minimum of the larger and smaller array's low water mark
values.

Note that the space overhead for referencing all the smaller arrays
when growing is relatively low: if we double the array size every time
we grow the array, the total overhead is less than the size of the
current array.

\minisec{Combining multiple shrinks}

Sometimes it is useful to combine multiple shrink operations, that is
to shrink back not to the previous smaller array, but to one (or more)
preceding it. For example, suppose that we had 5 growing operations:
$a_1 \rightarrow a_2 \rightarrow a_3 \rightarrow a_4 \rightarrow a_5$
(here $a_i$ represents an array, and $a_{i+1}$ is a bigger array than
$a_i$), and that on the next \lstinline!popBottom! operation we find
out that almost all the deque elements were stolen, and that the
number of elements left is less than some fraction of the size of
$a_1$. In such a case it makes more sense to shrink from $a_5$
directly to $a_1$, without going through all the intermediate
arrays. Extra caution should be taken, however, when choosing which
entries to copy using the low water mark: when copying from $a_5$ to
$a_1$ , the low water mark is the minimum of the low water marks of
$a_5$ and all the intermediate arrays (that is $a_4$, $a_3$ and
$a_2$).

\subsubsection{Working With a Shared Pool of Buffers}

The deque algorithm presented depends upon a garbage collector to
reclaim the unused buffers. For work-stealing algorithms, where each
process has its own deque and the maximum amount of memory needed by
all deques together can usually be bounded, it is often more suitable
to use the shared pool model.

With the shared pool model, the extra available buffers for all the
deques are kept in a shared pool. Whenever the deque's owner needs a
bigger array, it allocates one (of the appropriate size) from the
pool, and whenever it shrinks to a smaller array and does not need the
bigger array anymore, it can return it to the pool. There are two main
advantages for the shared pool model: First, it is much less expensive
to reclaim and allocate buffers from the shared pool than to allocate
them from the heap and use the garbage collector for
reclamation. Second, as described in this section, by assuming that
the reclaimed buffers are not returned to a global use by the
operating system, the deque's owner can reclaim a buffer while there
may be still some thieves referencing it (something that the garage
collector will not do), which leads to a better use of the allocated
space.

\minisec{Allocating and reclaiming buffers}

Whenever the deque's owner needs a new array it allocates the array's
buffer from the shared pool. Reclaiming buffers is trickier: Consider
the case in which a \lstinline!steal! operation executes Line 13, and
then a concurrent \lstinline!popBottom! operation shrinks the array
and returns the bigger array's buffer to the shared pool. When the
\lstinline!steal! operation resumes and executes Line 16, it reads and
returns an element from the old bigger array which was already
reclaimed, and therefore may be used by another deque. This scenario,
of course, could not be allowed. Therefore, to return an array's
buffer to the shared pool when shrinking back to a smaller one
\footnote{We assume that we do not reclaim the buffers of the smaller
 arrays when growing to bigger ones (for a faster shrink operation as
  described in Section 3.1). It is straightforward, however, to use
  the same solution for reclaiming buffers also when growing.}, the
deque's owner must have some guarantee that any thieves concurrently
referencing that array will be forced to abort and will not return
data read from it after it was reclaimed.

One way to do it is by self-stealing: the deque's owner steals an item
from its own deque and returns that item instead of an item popped
from the \lstinline!bottom! of the deque. This solution, however, is
problematic because it breaks the LIFO behavior that is normally
expected for the deque's owner.

Instead, we can use the property that the algorithm references the
array modulo its size, and increment \lstinline!top! and
\lstinline!bottom! in a way that aborts concurrent \lstinline!steal!
operations, but does not change the content of the deque. For a given
array size $N$, $x$ and $x + N$ both address the same element in the
array. Therefore, incrementing \lstinline!top! and \lstinline!bottom!
by the size of the new array when an array is shrunk or grown
preserves the deque's content, but aborts any concurrent thefts. In
Section 4.4 we show that modifying \lstinline!top! and
\lstinline!bottom! as described still allows us to assume that they
will not overflow.

The modified \lstinline!perhapsShrink! and \lstinline!steal! methods
appear in Figures 7 and 8. The new \lstinline!perhapsShrink! method
contains the \lstinline!bottom! and \lstinline!top! modifications
(Lines 39.2-39.6), right after making the new array active (Line
39.1), and before returning the old array's buffer to the shared pool
(Line 39.7). The order in which \lstinline!bottom! and \lstinline!top!
are modi- fied is important, because while the deque's owner is
between these modifications, thieves must not spuriously return a
value from an empty deque or report empty when elements are actually
in the deque. Because the new \lstinline!perhapsShrink! method
modifies \lstinline!bottom! before \lstinline!top!, a new empty deque
scenario is added, which is when the difference between
\lstinline!bottom! and \lstinline!top! is exactly the current array
size. Because the \lstinline!pushBottom! method never fills the array
to its maximum capacity, this difference uniquely identifies the new
empty scenario, and can be detected by the \lstinline!steal! method.

The CAS operation at Line 39.5. tries modifying \lstinline!top!. If it
fails, then there must have been a concurrent \lstinline!steal!
operation that modified \lstinline!top! after it was read by
\lstinline!perhapsShrink! at Line 39.4, at which point the smaller
array is already active. This implies that any subsequent CAS by
another concurrent \lstinline!steal! operation that read an element
from the old array will fail. Therefore if the CAS at Line 39.5 fails,
the \lstinline!perhapsShrink! method simply returns \lstinline!bottom!
to its original value (Line 39.6), and reclaims the old array buffer.

The \lstinline!steal! method is modified in two places: An additional
read of the active array is added (Line 11.1), and the emptiness check
is extended (Lines 15.1-15.4). The emptiness check must be extended
because the \lstinline!steal! operation may read \lstinline!bottom!
and \lstinline!top! values that correspond to a deque in the
``intermediate state'' between the \lstinline!bottom! and
\lstinline!top! updates of a shrink operation. These values may
indicate an empty deque although the difference between them is
possitive. Therefore, instead of only checking whether the difference
between \lstinline!bottom! and \lstinline!top! is non-positive, the
modified \lstinline!steal! method also checks whether this difference
modulo the array size is 0 (Line 15.1), for the array reference read
at Line 13.

If the test at Line 15.1 succeeds, and the values read for
\lstinline!top!, \lstinline!bottom!, and \lstinline!activeArray! at
Lines 11,12 and 13, correspond to some state of the memory, then this
state indicates an empty deque. To check whether these values indeed
correspond to an actual state of the memory, we use the additional
\lstinline!activeArray! reference read at Line 11.1. It can be shown
that if the values of both \lstinline!top! and \lstinline!activeArray!
were not modified from before reading \lstinline!bottom! at Line 12
until right after reading \lstinline!activeArray! at Line 13, then the
values read correspond to the memory state during the read of
\lstinline!bottom! at Line 12. Therefore, if the test at Line 15.2
succeeds, the method returns Empty \footnote{In the correctness proof
  we show that we do not need to worry about the ABA problem on the
  \lstinline!activeArray! value.}; otherwise, some concurrent
operation successfully popped an element from the deque
\footnote{Assuming the \lstinline!perhapsShrink! method is called only
  for a successful \lstinline!popBottom! operation.}, and the method
returns Abort.

Note that the deque's reference to the array is the first to be
updated by the \lstinline!perhapsShrink! method (Line 39.1), but the
last to be read by the \lstinline!steal! operation (Line 13), thereby
guaranteeing that if the \lstinline!steal! operation sees an
intermedi- ate state, it sees the new (smaller) array. In the full
version of the paper, we prove that the shared pool version of the
algorithm implements a linearizable ABP-style deque that cannot
overflow. Here we only informally show that using the above
\lstinline!perhapsShrink! and \lstinline!steal! methods, prevents a
\lstinline!steal! operation from returning an entry read from an array
after it was returned to the shared pool.

\begin{definition}
  Array States:
  \begin{itemize}
  \item A live array is an array whose buffer does not reside in the
    shared pool.
  \item The active array of a deque is the array referenced by the
    \lstinline!activeArray! data member of the deque object.
  \end{itemize}
\end{definition}

\begin{lemma}
  The \lstinline!steal! method never returns an entry that was read
  from an array which is not live.
\end{lemma}

\begin{proof}
  The only statement that returns buffers to the shared pool is Line
  39.7 in the \lstinline!perhapsShrink! method, which is executed only
  by the deque's owner. Also, the deque's owner is the only process
  that may activate or de-activate an array (that is, replace the
  active array of the deque). Since the \lstinline!perhapsShrink!
  method set the new smaller array to be the active array before
  executing Line 39.7, then an active array is never deallocated.

  The \lstinline!steal! operation returns an entry only if the CAS
  statement at Line 17 successfully writes the t + 1 value to
  \lstinline!top!, in which case it returns the entry read at Line
  16. The \lstinline!steal! operation reads the deque's active array
  at Line 13, and therefore the array is live at that point. Thus, if
  the \lstinline!steal! operation reads an entry from a non-live
  array, it must be the case that a concurrent
  \lstinline!perhapsShrink! operation:

  \begin{enumerate}
  \item Executes Line 39.7 before the \lstinline!steal! operation
    executes Line 16, and
  \item Executes Line 39.1 after the \lstinline!steal! operation
    executed Line 13.
  \end{enumerate}

  This implies that Line 39.5 is executed after the \lstinline!steal!
  operation reads \lstinline!top! at Line 11, but before it executes
  Line 17. If the CAS at Line 39.5 succeeds, then the CAS at Line 17
  fails. Otherwise, there must have been a concurrent
  \lstinline!steal! operation that modified \lstinline!top! after the
  \lstinline!perhapsShrink! executed Line 39.4, which also guarantees
  that the CAS at Line 17 fails. We can conclude then that if a
  \lstinline!steal! operation read an entry from a non-live array,
  then the CAS operation at Line 17 fails, which implies that the
  entry is not returned by that operation.
\end{proof}

\minisec{Lock freedom}

The algorithm we presented is clearly wait-free: the only loop in the
algorithm is the one that copies the data from one array to another,
and it is guaranteed to be finite (linear in the length of the array
from which we copy).

A more interesting property, which we formally prove in the full
version of the paper, is that even if the \lstinline!steal! operation
retries every time it returns Abort, the algorithm is
lock-free. Informally, this is correct because a \lstinline!steal!
operation only returns Abort when either \lstinline!top! is modified
by a concurrent pop operation (either a \lstinline!popBottom! or a
\lstinline!steal! operation), or that \lstinline!top! or
\lstinline!activeArray!  are modified by a \lstinline!perhapsShrink!
operation. Note that neither a pop nor a \lstinline!perhapsShrink!
operation can cause more than 2 \lstinline!steal! operations by the
same process to return Abort. Therefore, if a \lstinline!steal!
operation by a process returns Abort infinitely often, then there must
be other concurrent operations that complete successfully.

\minisec{Deallocating memory from the shared pool}

In case the shared pool runs out of buffers, it is possible to
allocate more memory to it. Freeing memory from the shared pool is a
more sensitive issue, especially in a strongly typed language, because
reuse of the freed buffers can cause differently-typed data to be
stored into these buffers. This may cause what is locally an actual
typing error (between lines 16 and 19), because there might be some
thieves that still reference the array that was stored in this buffer
(even in non-strongly typed languages, releasing memory to the
operating system and then referencing it may cause a runtime
error). Therefore in order to safely free an array's buffer from the
shared pool, we must first ensure that no thief is referencing that
array. This is done automatically by a garbage collector, if one
exists. Otherwise, other mechanisms should be used, like the Pass the
Buck algorithm by Herlihy, Luchangco and Moir [7]. It is important to
note, however, that freeing memory from the shared pool is a rare
operation (usually it is not needed at all), and therefore even using
a relatively expansive algorithm (like a garbage collector) should not
significantly hurt the performance of the algorithm.

\minisec{Analysis}

Shrinking in the shared pool model consumes deque indices more rapidly
than the earlier form of the algorithm, because every time the array
is shrunk, \lstinline!top! and \lstinline!bottom! are incremented by
the size of the smaller array. However, because of the policy of
shrinking only at a particular utilization, the consumption of indices
is still linear in the number of operations. Indices are consumed
fastest when the deque oscillates between containing $\frac{N}{2}$ and
$\frac{N}{K}$ elements, where N is the capacity of the larger of two
arrays and $\frac{1}{K}$ is the utilization fraction under which we
shrink the array. Each time the deque grows to $\frac{N}{2}$ elements,
it expands out of the small array; each time it reduces to
$\frac{N}{K}$ elements, it shrinks, and the \lstinline!top! and
\lstinline!bottom! indices are bumped for- ward by $\frac{N}{2}$. The
number of operations in a grow-shrink cycle is $2N \times (\frac{1}{2}
- \frac{1}{K})$. Growth of \lstinline!top! is maximized if the
operations that remove the elements from the deque are steals. The
growth in one worst-case grow-shrink cycle is \lstinline!top! is $N
\times (\frac{1}{2} - \frac{1}{K}) + \frac{N}{2}$. Dividing,
cancelling, and simplifying produces a worst-case growth of $(K -
1)/(K - 2)$ indices per operation, which is small given that $K \ge
3$.

If $K$ is $3$, then the amortized index consumption of each operation
is no larger than $2$. A 64-bit index will not overflow until at least
263 operations are performed. If operations occur at a rate of 232 per
second, this provides at least 231 seconds of run-time. One year is
roughly equal to 225 seconds, so a 64 bit index permits at least 64
years of continuous operation without overflow.

Because the active array in the deque is at least $\frac{1}{K}$
filled, and because the total size of the smaller blocks is at most
the size of the active array, the storage overhead of this algorithm
is at most $2K$ times the current deque size \footnote{This upper
  bound assumes that the \lstinline!perhapsShrink! method is called by
  the deque's owner often enough to shrink the deque’s array when
  necessary.}.

\subsubsection{Performance}

We evaluated the performance of the new dynamic circular work-stealing
algorithm in comparison to the original fixed-array ABP work-stealing
algorithm. We implemented both algorithms in C++, and used a simple
shared pool algorithm that allocates and frees a buffer with a single
CAS instruction.

The benchmark we ran simulates load balancing of a general computation
by building the DAG corresponding to the computation
\cite{Blumofe1999}, as follows: Initially a single deque contains a
single node representing the first work item of the
computation. Processes pop nodes from their own deques, and
\lstinline!steal! nodes from other deques if their own deque is
empty. Each time a process pops a node from a deque, it generates up
to B child nodes, and pushes them into its deque (B represents the
maximum branch of the DAG, and it is a configurable parameter). The
number of child nodes generated for a node is randomly chosen with
probability that is inversely proportional to the depth of that node
in the DAG. The expected number of child nodes for a node of depth $d$
in a DAG of maximum depth $D$ is: $B \cdot \left(1 - \frac{d}{D}
\right)$. To get the most accurate measure of the performance
difference between the two algorithms, we did not perform any work on
a node other than pushing its child nodes to the process's deque.

We ran the benchmark on a 16 node Sun EnterpriseTM 6500, an SMP
machine formed from 8 boards of two 400MHz UltraSparc processors,
connected by a crossbar UPA switch, and running a SolarisTM 9
operating system. We chose the maximum branch of the DAG to be 13, and
the maximum depth to be 10. We used 72-element arrays for the original
ABP deques, and our algorithm allocated the deques with an initial
size of 64-elements, plus a few 128-element arrays in the shared pool.

Figure 9 presents the throughput of both algorithms, running
stand-alone, as a function of the number of processes. As can be
seen, both algorithms scale well, and the performance difference is
relatively small. Recall that our benchmark does not perform any real
computation -- it only measures the load-balancing algorithm
overhead. In real applications the time spent on the load-balancing
algorithm Next we ran our benchmark in a multiprogrammed fashion by
running multiple instances of it in parallel, where each instance is
running with 16 processes (as the number of processors on the
machine). Figure 10 presents the throughput of an instance as a
function the multiprogramming level. As can be seen, there is no
significant difference in the performance of the two algorithms.

To compare the stability of the two algorithms, we measured how many
of our 64-element arrays overflowed and needed a 128-element array
from the shared pool, and noticed that at most one 128-element array
is ever needed. On the other hand, the 72-element array allocated for
each of the deques in the original ABP algorithm was not always
sufficient, and in some cases the algorithm failed to complete due to
an overflow of an array. These failures became more frequent as the
level of multiprogramming increased. Therefore, for the same amount of
array space (notice that $16 \cdot 72 = 128 + (16 \cdot 64)$), we get
more robustness with our new algorithm than with the original ABP
algorithm, without a noticeable cost in performance.

\subsubsection{Summary}

We present the first ABP-style circular work stealing deque that
cannot overflow. Our algorithm is simple, its space complexity is
linear in the number of elements in the deque, and it does not require
a garbage collector for its memory management. It may be interesting
to see how our techniques are applied to other schemes that improve on
ABP-work stealing such as the locality-guided work-stealing of Acar,
Blelloch and Blumofe \cite{Acar2002} or the steal-half algorithm of
Hendler and Shavit \cite{Hendler2002}.


\subsection{The art of multiprocessor programming \cite{Herlihy2008}}

A limitation of the \lstinline!BoundedDEQueue! class is that the queue
has a fixed size. For some applications, it may be difficult to
predict this size, especially if some threads create significantly
more tasks than others. Assigning each thread its own
\lstinline!BoundedDEQueue! of maximal capacity wastes space.

To address these limitations, we now consider an unbounded
double-ended queue \lstinline!UnboundedDEQueue! class that dynamically
resizes itself as needed.

We implement the \lstinline!UnboundedDEQueue! in a cyclic array, with
\lstinline!top! and \lstinline!bottom! fields as in the
\lstinline!BoundedDEQueue! (except indexed modulo the array's
capacity). As before, if \lstinline!bottom! is less than or equal to
\lstinline!top!, the \lstinline!UnboundedDEQueue! is empty. Using a
cyclic array eliminates the need to reset \lstinline!bottom! and
\lstinline!top! to \lstinline!0!. Moreover, it permits \lstinline!top!
to be incremented but never decremented, eliminating the need for
\lstinline!top! to be an \lstinline!AtomicStampedReference!. Moreover,
in the \lstinline!UnboundedDEQueue! algorithm, if
\lstinline!pushBottom()! discovers that the current circular array is
full, it can resize (enlarge) it, copying the tasks into a bigger
array, and pushing the new task into the new (larger) array. Because
the array is indexed modulo its capacity, there is no need to update
the \lstinline!top! or \lstinline!bottom! fields when moving the
elements into a bigger array (although the actual array indexes where
the elements are stored might change).

The \lstinline!CircularTaskArray()! class is depicted in
Fig. 16.13. It provides \lstinline!get()! and put() methods that add
and remove tasks, and a \lstinline!resize()! method that allocates a
new circular array and copies the old array's contents into the new
array. The use of modular arithmetic ensures that even though the
array has changed size and the tasks may have shifted positions,
thieves can still use the \lstinline!top! field to find the next task
to steal.

The \lstinline!UnboundedDEQueue! class has three fields: tasks,
\lstinline!bottom!, and \lstinline!top! (Fig. 16.14, Lines 3–5). The
\lstinline!popBottom()! (Fig. 16.14) and \lstinline!popTop()! methods
(Fig. 16.15) are almost the same as those of the BoundedDEQueue, with
one key difference: the use of modular arithmetic to compute indexes
means the \lstinline!top! index need never be decremented. As noted,
there is no need for a timestamp to prevent ABA problems. Both
methods, when competing for the last task, steal it by incrementing
\lstinline!top!. To reset the \lstinline!UnboundedDEQueue! to empty,
simply increment the \lstinline!bottom! field to equal
\lstinline!top!. In the code, \lstinline!popBottom()!, immediately
after the compareAndSet() in Line 27, sets \lstinline!bottom! to equal
\lstinline!top+1! whether or not the compareAndSet() succeeds,
because, even if it failed, a concurrent thief must have stolen the
last task. Storing \lstinline!top+1! into \lstinline!bottom! makes
\lstinline!top! and \lstinline!bottom! equal, resetting the
\lstinline!UnboundedDEQueue! object to an empty state.

The isEmpty() method (Fig. 16.14) first reads \lstinline!top!, then
\lstinline!bottom!, checking whether \lstinline!bottom! is less than
or equal to \lstinline!top! (Line 4). The order is important because
\lstinline!top! never decreases, and so if a thread reads
\lstinline!bottom! after \lstinline!top! and sees it is no greater,
the queue is indeed empty because a concurrent modification of
\lstinline!top! could only have increased the \lstinline!top!
value. The same principle applies in the \lstinline!popTop()! method
call. An example execution is provided in Fig. 16.16.

The \lstinline!pushBottom()! method (Fig. 16.14) is almost the same as
that of the BoundedDEQueue. One difference is that the method must
enlarge the circular array if the current push is about to cause it to
exceed its capacity. Another is that \lstinline!popTop()! does not
need to manipulate a timestamp. The ability to resize carries a price:
every call must read \lstinline!top! (Line 21) to determine if a
resize is necessary, possibly causing more cache misses because
\lstinline!top! is modified by all processes. We can reduce this
overhead by having threads save a local value of \lstinline!top! and
using it to compute the size of the \lstinline!UnboundedDEQueue!
object. A thread reads the \lstinline!top!field only when this bound
is exceeded, indicating that a \lstinline!resize()! may be necessary.
Even though the local copy may become outdated because of changes to
the shared \lstinline!top!, \lstinline!top! is never decremented, so
the real size of the \lstinline!UnboundedDEQueue! object can only be
smaller than the one calculated using the local variable.

In summary, we have seen two ways to design a nonblocking linearizable
\lstinline!DEQueue! class. We can get away with using only loads and
stores in the most common manipulations of the DEQueue, but at the
price of having more complex algorithms. Such algorithms are
justifiable for an application such as an executer pool whose
performance may be critical to a concurrent multithreaded system.

\lstinputlisting[style=Numbers,
  caption={Work-stealing deque}, 
  label=lst:work-stealing-deque]{
    ../listings/queues-implementation/WorkStealingDeque.java
}


\section{Dynamic Work-Stealing Deque}
\label{sec:queues-implementation-dynamic-ws-deque}

\begin{itemize}
\item A dynamic-sized nonblocking work stealing deque
  \cite{Hendler2006}
\item A dynamic-sized nonblocking work stealing deque
  \cite{Hendler2006a}
\end{itemize}

\todo{Finish section ``Dynamic Work-Stealing Deque''}

\subsubsection{Dynamic circular work-stealing deque \cite{Chase2005}}

The list-based work-stealing deque algorithm presented by Hendler, Lev
and Shavit \cite{Hendler2006, Hendler2006a} uses a list of small
arrays to eliminate the overflow problem. However, it is relatively
complicated, does not use cyclic arrays (and therefore wastes some
memory), and introduces a trade-off between its time and space
complexity due to the extra work required for the list's maintenance.

\lstinputlisting[style=Numbers,
  caption={Dynamic work-stealing deque}, 
  label=lst:dynamic-work-stealing-deque]{
    ../listings/queues-implementation/DynamicWorkStealingDeque.java
}


\section{Duplicating Work-Stealing Deque}
\label{sec:queues-implementation-duplicating-ws-deque}

\begin{itemize}
\item The design of a task parallel library \cite{Leijen2009}
\item Checkfence: checking consistency of concurrent data types on
  relaxed memory models \cite{Burckhardt2007}
\item Checkfence: checking consistency of concurrent data types on
  relaxed memory models \cite{Burckhardt2007a}
\end{itemize}

\todo{Boolean ``Already Run'' could be helpful when using a mailbox
  style implementation for locality-aware scheduling \cite{Acar2002}}

\todo{Finish section ``Duplicating Work-Stealing Deque''}

\lstinputlisting[style=Numbers,
  caption={Interval}, 
  label=lst:interval]{
    ../listings/queues-implementation/Interval.java
}

\lstinputlisting[style=Numbers,
  caption={Duplicating work-stealing deque}, 
  label=lst:work-stealing-deque]{
    ../listings/queues-implementation/DuplicatingWorkStealingDeque.java
}


\section{Idempotent Work-Stealing Deque}
\label{sec:queues-implementation-idempotent-ws-deque}

\begin{itemize}
\item Idempotent work stealing \cite{Michael2009}
\end{itemize}

\todo{Finish section ``Idempotent Work-Stealing Deque''}

\lstinputlisting[style=Numbers,
  caption={Idempotent work-stealing deque}, 
  label=lst:idempotent-work-stealing-deque]{
    ../listings/queues-implementation/IdempotentWorkStealingDeque.java
}


%%% Local Variables: 
%%% mode: latex
%%% TeX-master: "thesis"
%%% End: 
