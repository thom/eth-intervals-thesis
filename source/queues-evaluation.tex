%==============================================================================
% queues-evaluation.tex
%==============================================================================

\chapter{Evaluation}
\label{chap:queues-evaluation}

\todo[inline]{Finish chapter ``Evaluation''}

\begin{itemize}
\item Benchmarks
\item Idempotent FIFO and LIFO queue, deque; duplicating queues
\item Global shared work queue
\item $\le 8$ cores
\end{itemize}

\section{References}
\subsection{Dynamic circular work-stealing deque \cite{Chase2005}}

We evaluated the performance of the new dynamic circular work-stealing
algorithm in comparison to the original fixed-array ABP work-stealing
algorithm. We implemented both algorithms in C++, and used a simple
shared pool algorithm that allocates and frees a buffer with a single
CAS instruction.

The benchmark we ran simulates load balancing of a general computation
by building the DAG corresponding to the computation
\cite{Blumofe1999}, as follows: Initially a single deque contains a
single node representing the first work item of the
computation. Processes pop nodes from their own deques, and
\lstinline!steal! nodes from other deques if their own deque is
empty. Each time a process pops a node from a deque, it generates up
to B child nodes, and pushes them into its deque (B represents the
maximum branch of the DAG, and it is a configurable parameter). The
number of child nodes generated for a node is randomly chosen with
probability that is inversely proportional to the depth of that node
in the DAG. The expected number of child nodes for a node of depth $d$
in a DAG of maximum depth $D$ is: $B \cdot \left(1 - \frac{d}{D}
\right)$. To get the most accurate measure of the performance
difference between the two algorithms, we did not perform any work on
a node other than pushing its child nodes to the process's deque.

We ran the benchmark on a 16 node Sun Enterprise 6500, an SMP machine
formed from 8 boards of two 400MHz UltraSparc processors, connected by
a crossbar UPA switch, and running a SolarisTM 9 operating system. We
chose the maximum branch of the DAG to be 13, and the maximum depth to
be 10. We used 72-element arrays for the original ABP deques, and our
algorithm allocated the deques with an initial size of 64-elements,
plus a few 128-element arrays in the shared pool.

Figure 9 presents the throughput of both algorithms, running
stand-alone, as a function of the number of processes. As can be
seen, both algorithms scale well, and the performance difference is
relatively small. Recall that our benchmark does not perform any real
computation -- it only measures the load-balancing algorithm
overhead. In real applications the time spent on the load-balancing
algorithm Next we ran our benchmark in a multiprogrammed fashion by
running multiple instances of it in parallel, where each instance is
running with 16 processes (as the number of processors on the
machine). Figure 10 presents the throughput of an instance as a
function the multiprogramming level. As can be seen, there is no
significant difference in the performance of the two algorithms.

To compare the stability of the two algorithms, we measured how many
of our 64-element arrays overflowed and needed a 128-element array
from the shared pool, and noticed that at most one 128-element array
is ever needed. On the other hand, the 72-element array allocated for
each of the deques in the original ABP algorithm was not always
sufficient, and in some cases the algorithm failed to complete due to
an overflow of an array. These failures became more frequent as the
level of multiprogramming increased. Therefore, for the same amount of
array space (notice that $16 \cdot 72 = 128 + (16 \cdot 64)$), we get
more robustness with our new algorithm than with the original ABP
algorithm, without a noticeable cost in performance.


%%% Local Variables: 
%%% mode: latex
%%% TeX-master: "thesis"
%%% End: 
