%==============================================================================
% locality-introduction.tex
%==============================================================================

\part{Locality-Aware Work-Stealing}
\label{part:locality}

\chapter{Introduction}
\label{chap:locality-introduction}

The current implementation of the intervals library uses a
locality-ignorant work-stealing scheduler to schedule ready-to-run
tasks. In this thesis we introduce LASSI, a locality-aware scheduler
for intervals.\footnote{The correct acronym would be LASI but we chose
  LASSI instead as we really enjoy drinking refreshing masala lassi
  \smiley}

\section{Locality-Aware Intervals Scheduling}
\label{sec:locality-introduction-locality-aware-intervals-scheduling}

In chip multiprocessor systems it may be more efficient to schedule a
task on one processor than another. As modern CMPs feature a
heterogeneous memory hierarchy where access times depend on which
processor an interval is running, locality-aware intervals can lead to
improved performance:

\begin{itemize}
\item By scheduling data sharing intervals on the same processor they
  perform prefetching of shared regions for one another.
\item Scheduling non-communicating intervals with high memory
  footprints on different processors helps to reduce cache contention
  and potential cache capacity problems.
\end{itemize}

When using locality-ignorant work-stealing we cannot fully exploit the
heterogeneous memory hierarchy of CMPs for our benefit. Thus, we
implement and analyze LASSI, a locality-aware scheduler for
intervals. LASSI is designed for locality-aware scheduling using
locality hints provided by the programmer. Instead of employing
work-stealing workers, it groups workers into \emph{Work-Stealing
  Places}.

Each work-stealing place has a fixed number of workers and a local
deque to maintain ready tasks. The workers of a place share its local
deque from which they obtain work. When a worker finds that the pool
of its place is empty, it tries to steal a task from the pool of a
victim place chosen at random. Locality-aware intervals are added to
their preferred place once they are ready for scheduling.

Providing locality hints to intervals is optional and the performance
of locality-ignorant programs executed with the new scheduler
implementation is comparable to that of the original scheduler.

Our experimental results show that \emph{best locality} placement of
intervals can achieve up to 1.15\texttimes\ speedup over \emph{worst}
or \emph{ignorant locality} placement. Cache hits can be increased by
up to 1.5\texttimes\ and cache misses can be reduced by up to
3.1\texttimes\ for the benchmarks and platform studied in this thesis.

\todo{Correct speedups, and cache hits and misses}


\section{Overview}
\label{sec:locality-introduction-locality-overview}

Chapter \ref{chap:locality-approach} describes our approach in
evaluating locality-aware scheduling. Chapter
\ref{chap:locality-implementation} explains the implementation of
LASSI. The chapter presents the locality-aware intervals API,
introduces \emph{Work-Stealing Places}, and shows how worker threads
are bound to specific processing units, e.g. cores.  In Chapter
\ref{chap:locality-performance} we describe the locality benchmarks
and analyze their results. Chapter \ref{chap:locality-related-work}
puts our research in the context of related work. In Chapter
\ref{chap:locality-conclusions-and-future-work} we conclude and
summarize our research, and give some ideas for future work.


%%% Local Variables: 
%%% mode: latex
%%% TeX-master: "thesis"
%%% End: 
