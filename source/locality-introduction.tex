%==============================================================================
% locality-introduction.tex
%==============================================================================

\part{Locality-Aware Work-Stealing}
\label{part:locality}

\chapter{Introduction}
\label{chap:locality-introduction}

\todo[inline]{Structure of part II}

\section{Problem and Motivation}
\label{sec:locality-intro-problem-and-motivation}

\minisec{Brief announcement: parallel depth first vs. work stealing
  schedulers on CMP architectures \cite{Liaskovitis2006}}

In the work-stealing scheduler the interval implementation is using,
each processing core maintains a local work double-ended queue (deque)
of ready-to-execute intervals. Whenever its local deque is empty, the
core steals an interval from the bottom of the first non-empty deque
it finds. Work-stealing is an attractive scheduling policy because
when there is plenty of parallelism, stealing is quite rare. However,
work-stealing is not designed for constructive cache sharing, because
the cores tend to have disjoint working sets.

\minisec{Scheduling threads for constructive cache sharing on CMPs
  \cite{Chen2007}}

Work Stealing (WS) is a popular greedy thread scheduling algorithm
\footnote{In a greedy schedule, a ready job remains unscheduled only
  if all processors are already busy.} for programs, with proven
theoretical properties with regards to memory and cache usage
\cite{Blumofe1998a, Blumofe1999, Acar2002}. The policy maintains a
work queue for each processor (actually a double-ended queue which
allows elements to be inserted on one end of the queue, the top, but
taken from either end). When forking a new thread, this new thread is
placed on the top of the local queue. When a thread completes on a
processor, the processor looks for a ready-to-execute thread by first
looking on the top of the local queue. If it finds a thread, it takes
the thread off the queue and runs it. If the local queue is empty it
checks the work queues of the other processors and steals a thread
from the bottom of the first non-empty queue it finds. WS is an
attractive scheduling policy because when there is plenty of
parallelism, stealing is quite rare and, because the threads in a
queue are related, there is good affinity among the threads executed
by any one processor. However, WS is not designed for constructive
cache sharing, because the processors tend to have disjoint working
sets.


\todo{Finish section ``Problem and Motivation''}

\section{Aim}
\label{sec:locality-intro-aim}

\todo[inline]{Finish section ``Aim''}

\section{Overview}
\label{sec:locality-intro-overview}

\todo[inline]{Finish section ``Overview''}

%%% Local Variables: 
%%% mode: latex
%%% TeX-master: "thesis"
%%% End: 
