%==============================================================================
% abstract.tex
%==============================================================================

\chapter*{Abstract}
\label{chap:abstract}

Intervals are a new, higher-level primitive for parallel programming
which permits programmers to directly construct the program
schedule. They are under active development at ETH Zürich as part of
the PhD research of Nicholas D. Matsakis.

The intervals implementation in Java uses a work-stealing scheduler
where a worker running out of work tries to ``steal'' work from
others. The scope of this thesis is to improve the performance of the
intervals scheduler.

We implement and analyze locality-aware scheduling of
intervals. Locality-aware scheduling allows each interval to be given
an affinity for a place, and when a worker belonging to a certain
place obtains an interval, it gives priority to the intervals with
affinity for the place.

\begin{center}
  $\bullet$
\end{center}

The performance of work-stealing schedulers is in a large part
determined by the efficiency of their work queue implementations. In
the non-blocking work-stealing scheduler \cite{Arora2001}, the deques
are implemented with non-blocking synchronization. That is, instead of
using mutual exclusion, its deques use atomic synchronization
primitives such as Compare-and-Swap. The current deque implementation
of intervals however uses mutual exclusion when trying to steal. Thus,
as a separate effort, we design and explore alternative non-blocking
queue implementations with the aim to improve work-stealing
performance.


%%% Local Variables: 
%%% mode: latex
%%% TeX-master: "thesis"
%%% End: 