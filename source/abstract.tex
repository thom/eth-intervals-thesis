%==============================================================================
% abstract.tex
%==============================================================================

\chapter*{Abstract}
\label{chap:abstract}

Intervals are a new, higher-level primitive for parallel programming
allowing programmers to directly construct the program schedule. They
are under active development at ETH Zürich as part of the PhD research
of Nicholas D. Matsakis.

The intervals implementation in Java uses a work-stealing scheduler
where a worker running out of work tries to ``steal'' work from
others. The scope of this thesis is to improve the performance of the
intervals scheduler.

We implement and analyze an advanced scheduler for intervals. It is
designed for locality-aware scheduling using locality hints provided
by the programmer. Instead of employing work-stealing workers, our
scheduler groups workers into \emph{Work-Stealing Places}. Each
work-stealing place has a fixed number of workers and a local deque to
maintain ready tasks. The workers of a place share its local deque
from which they obtain work. When a worker finds that the pool of its
place is empty, it tries to steal a task from the pool of a victim
place chosen at random. Locality-aware intervals are added to their
preferred place once they are ready for scheduling. 

Providing locality hints to intervals is optional and the performance
of locality-ignorant programs executed with the new scheduler
implementation is comparable to that of the original scheduler.

\begin{center}
  $\bullet$
\end{center}

The performance of work-stealing schedulers is in a large part
determined by the efficiency of their work queue implementations. In
the non-blocking work-stealing scheduler \cite{Arora1998}, the queues
are implemented with non-blocking synchronization. That is, instead of
using mutual exclusion, it uses atomic synchronization primitives such
as Compare-and-Swap. The current work-stealing queue of intervals
however uses mutual exclusion when trying to steal. Thus, as a
separate effort, we design and explore alternative non-blocking queue
implementations with the aim to improve work-stealing performance.


%%% Local Variables: 
%%% mode: latex
%%% TeX-master: "thesis"
%%% End: 