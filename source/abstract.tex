%==============================================================================
% abstract.tex
%==============================================================================

\chapter*{Abstract}
\label{chap:abstract}

Intervals \cite{Matsakis2009b} are a new, higher-level primitive for
parallel programming with which programmers directly construct the
program schedule. They are under active development at ETH Zürich as
part of the PhD research of Nicholas Matsakis \cite{Matsakis2010}.

The intervals implementation in Java uses a work-stealing scheduler in
which a worker that runs out of work tries to ``steal'' work from
others. The scope of this thesis is to improve the efficiency of the
intervals scheduler.

In this thesis we implemented and analyzed locality-aware scheduling of
intervals. Locality-aware scheduling allows each interval to be given
an affinity for a place, and when a worker belonging to a certain
place obtains an interval, it gives priority to the intervals with
affinity for the place \cite{Acar2002, Guo2010}.

\begin{center}
  $\bullet$
\end{center}

The original work-stealing algorithm uses non-blocking algorithms to
implement queue operations \cite{Arora2001}. However, the current
deque implementation of intervals uses a lock when trying to steal. As
a separate effort, we explored and designed alternative non-blocking
queue implementations with the aim to improve scheduling performance.

\todo{Rewrite abstract}

% Intervals are a new, higher-level primitive for parallel programming
% with which programmers directly construct the program schedule. They
% are under active development at ETH Zürich as part of the PhD research
% of Nicholas Matsakis \cite{Matsakis2010}.

% The intervals implementation in Java uses a work-stealing scheduler in
% which a worker that runs out of work tries to ``steal'' work from
% others. The goal of this thesis is to improve the efficiency of the
% intervals scheduler.

% \subsubsection*{\autoref{part:locality}. Locality-Aware Work-Stealing}

% Currently the intervals scheduler randomly schedules work items and
% when stealing a work item, the worker chooses his victim by random. To
% improve performance and reduce cache misses, work items that access
% the same data should be scheduled on the same worker or a worker
% ``nearby''.

% In the second part of the thesis we implemented and analyzed
% locality-aware scheduling of intervals. In locality-aware scheduling,
% each interval can be given an affinity for a place, and when a worker
% belonging to a certain place obtains an interval, it gives priority to
% the intervals with affinity for the place.

% \subsubsection*{\autoref{part:queues}. Work-Stealing Queues}

% In a work-stealing scheduler, each worker keeps a pool of work items
% waiting to be executed. The current intervals implementation uses
% doubled-ended queues (deque).  

% In the first part of this thesis we explored alternative queue
% implementations and compared them to the current implementation.


%%% Local Variables: 
%%% mode: latex
%%% TeX-master: "thesis"
%%% End: 