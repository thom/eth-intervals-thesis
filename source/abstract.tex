%==============================================================================
% abstract.tex
%==============================================================================

\chapter*{Abstract}
\label{chap:abstract}

Intervals are a new, higher-level primitive for parallel programming
with which programmers directly construct the program schedule. They
are under active development at ETH Zürich as part of the PhD research
of Nicholas Matsakis \cite{Matsakis2010}.

The intervals implementation in Java uses a work-stealing scheduler in
which a worker that runs out of work tries to ``steal'' work from
others. The goal of this thesis is to improve the efficiency of the
intervals scheduler.

\minisec{Part I. Explore and profile different work-stealing queue
  implementations}

In a work-stealing scheduler, each worker keeps a pool of work items
waiting to be executed. The current intervals implementation uses
doubled-ended queues (deque).  

In the first part of this thesis we explored alternative queue
implementations and compared them to the current implementation.

\minisec{Part II. Locality-aware work-stealing}

Currently the intervals scheduler randomly schedules work items and
when stealing a work item, the worker chooses his victim by random. To
improve performance and reduce cache misses, work items that access
the same data should be scheduled on the same worker or a worker
``nearby''.

In the second part of the thesis we implemented and analyzed
locality-aware scheduling of intervals. In locality-aware scheduling,
each interval can be given an affinity for a place, and when a worker
belonging to a certain place obtains an interval, it gives priority to
the intervals with affinity for the place.

%%% Local Variables: 
%%% mode: latex
%%% TeX-master: "thesis"
%%% End: 