%==============================================================================
% abstract.tex
%==============================================================================

\chapter*{Abstract}
\label{chap:abstract}

Intervals are a new, higher-level primitive for parallel programming
allowing programmers to directly construct the program schedule. They
are under active development at ETH Zürich as part of the PhD research
of Nicholas D. Matsakis.

The intervals implementation in Java uses a work-stealing scheduler
where a worker running out of work tries to ``steal'' work from
others. The scope of this thesis is to improve the performance of the
intervals scheduler.

We implement and analyze an advanced scheduler for intervals. It is
designed for locality-aware scheduling using locality hints provided
by the programmer. Instead of employing work-stealing workers, our
scheduler groups workers into \emph{Work-Stealing Places}.  Each
work-stealing place has a fixed number of workers and a local deque to
maintain ready tasks. The workers of a place share its local deque
from which they obtain work. When a worker finds that its place's pool
is empty, it becomes a thief and steals a task from the pool of a
victim place chosen at random. When an interval with affinity for a
place is ready for scheduling, it gets added to the place it has
affinity for.

\begin{center}
  $\bullet$
\end{center}

The performance of work-stealing schedulers is in a large part
determined by the efficiency of their work queue implementations. In
the non-blocking work-stealing scheduler \cite{Arora1998}, the queues
are implemented with non-blocking synchronization. That is, instead of
using mutual exclusion, it uses atomic synchronization primitives such
as Compare-and-Swap. The current work-stealing queue of intervals
however uses mutual exclusion when trying to steal. Thus, as a
separate effort, we design and explore alternative non-blocking queue
implementations with the aim to improve work-stealing performance.


%%% Local Variables: 
%%% mode: latex
%%% TeX-master: "thesis"
%%% End: 